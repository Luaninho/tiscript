\RequirePackage[2020-02-02]{latexrelease}
\documentclass[
    12pt,
    a4paper,
    table
]{book}
%DEBUG SETTINGS
%\usepackage{showframe}% http://ctan.org/pkg/showframe
% END DEBUG SETTINGS


\newif\ifusechapterhead
\usechapterheadtrue

\usepackage{amsmath}
\usepackage{geometry}
\geometry{
    headheight=27.19pt
}
\usepackage{graphicx}
\usepackage{ngerman}
\PassOptionsToPackage{hyphens}{url}
\usepackage[hidelinks]{hyperref}
\hypersetup {
    pdfpagemode = {UseNone},
    pdftitle = {Theoretische Informatik für Problemlöser:innen},
    pdfauthor = {Dr. Tobias Weber},
    pdflang = {en-US}
}

\hyphenation{Ent-scheid-ungs-hilfe}

% Referencing stuff
\usepackage{nameref} % referencing chapter names (provides \nameref{})
\renewcommand{\figureautorefname}{Abbildung}
\renewcommand{\tableautorefname}{Tabelle}
\renewcommand{\chapterautorefname}{Kapitel}
\renewcommand{\sectionautorefname}{Abschnitt}
\renewcommand{\subsectionautorefname}{Unterabschnitt}
\usepackage[utf8]{inputenx}
\usepackage[
backend=biber,
style=alphabetic,
citestyle=alphabetic,
maxbibnames=99
]{biblatex}
\addbibresource{bibliographie.bib}
\newcommand\citeallnames{\AtNextCite{\defcounter{maxnames}{99}}\citename}

\usepackage{catoptions} % \Autoref commands
%\usepackage[acronym,nopostdot]{glossaries}
%\makeglossaries
%\loadglsentries{content/glossary} % Conclusion
%\makeindex

% tables
\usepackage{float} % keep tables in sections
\usepackage{amssymb} 
\usepackage{array} % for custom table columns

\usepackage{titlesec} % for customizing headings



% Drop space before chapter headings
\usepackage{etoolbox}% http://ctan.org/pkg/etoolbox
\makeatletter
\patchcmd{\@makechapterhead}{\vspace*{50\p@}}{}{}{}% Removes space above \chapter head
\patchcmd{\@makeschapterhead}{\vspace*{50\p@}}{}{}{}% Removes space above \chapter* head
\makeatother
% end drop space before chapter headings

\usepackage{gitinfo2}

\usepackage{listings}
\definecolor{codegreen}{rgb}{0,0.6,0}
\definecolor{codegray}{rgb}{0.5,0.5,0.5}
\definecolor{codepurple}{rgb}{0.58,0,0.82}
\definecolor{backcolour}{rgb}{0.95,0.95,0.92}

\lstdefinestyle{mystyle}{
    backgroundcolor=\color{backcolour},   
    commentstyle=\color{codegreen},
    keywordstyle=\color{magenta},
    numberstyle=\tiny\color{codegray},
    stringstyle=\color{codepurple},
    basicstyle=\ttfamily\footnotesize,
    breakatwhitespace=false,         
    breaklines=true,                 
    captionpos=b,                    
    keepspaces=true,                 
    numbers=left,                    
    numbersep=5pt,                  
    showspaces=false,                
    showstringspaces=false,
    showtabs=false,                  
    tabsize=2
}
\lstset{style=mystyle}
% Automaten
\usepackage{tikz}
\usetikzlibrary{automata, positioning, arrows, arrows.meta}
\tikzset{
	->, % makes the edges directed
	>=stealth', % makes the arrow heads bold
	node distance=3cm, % specifies the minimum distance between two nodes. Change if necessary.
	every state/.style={thick, fill=gray!10}, % sets the properties for each ’state’ node
	initial text=$ $, % sets the text that appears on the start arrow
    every edge/.style = {draw, semithick},
}

\usepackage{enumitem} % für alphanumerische enumerates

% Zeilenumbrüche in Tabellenfeldern
\newcommand{\specialcell}[2][c]{%
  \begin{tabular}[#1]{@{}c@{}}#2\end{tabular}}


\setcounter{section}{-1}
\setcounter{tocdepth}{1}

\title{Theoretische Informatik für Problemlöser:innen
\\
\href{https://github.com/tgweber/tiscript/commit/\gitHash}{\gitHash}}
\author{Tobias Weber} 


%%%%%%%%%%%%%%%%%%%%%%%%%%%%
%%  Beginn des Dokuments  %%
%%%%%%%%%%%%%%%%%%%%%%%%%%%%
\begin{document}


  \frontmatter
\date{\gitAuthorIsoDate\\
Bitte Tippfehler und Korrekturvorschläge via github einkippen:
\url{https://github.com/tgweber/tiscript}
}
  \maketitle

  \include{content/remark}
  \chapter{Vorwort}

Warum braucht es eine weitere (deutschsprachige) Einführung
in die theoretische Informatik?
Gute Einführungen zu dem Thema füllen bereits Bücherregale
(siehe Ende dieses Vorwortes).
Mit einem Schmunzeln könnte man sagen,
dass das Alleinstellungsmerkmal dieses Skriptes seine
\emph{Unvollständigkeit} ist.
Im Gegensatz zu anderen Einführungen wollen wir das Fach nicht in der Breite darstellen,
sondern uns einen Gedankengang, eine spezifische Überlegung,
ein \emph{Narrativ} herauspicken:
Wir wollen die Frage beantworten,
warum diejenigen IT-Fachkräfte Probleme
schneller, effizienter und präziser lösen,
die das Kompendium der theoretischen Informatik beherrschen.
Die Kapitel 1 bis 6 umfassen genau diese Überlegung:

In Kapitel 1 wird zunächst genauer analysiert,
was überhaupt ein Problem im Sinne der theoretischen Informatik ist
und wie wir Probleme durch Abstraktion und Formalisierung mit den Werkzeugen
dieses Faches bearbeiten können.
Wir zeigen an einem simplen Beispiel,
wie ein Problem als formale Sprachen kodiert werden kann.

In Kapitel 2 beschäftigen wir uns mit dem Aufwand,
den diese Probleme, beziehungsweise deren Berechnungen verursachen.
Wir differenzieren zwischen Problem, Algorithmus und Implementierung,
fassen Aufwand genauer als benötigten (Speicher-)Platz oder Laufzeit.
Das Wachstum von Speicherplatz und Laufzeit lässt sich
mit der Landau-Notation quantitativ fassen. 
Die Diskussion um Grenzen und Vorteile dieser Methode
gegenüber empirischen Messungen runden das Kapitel ab.

In Kapitel 3 kehren wir zu den formalen Sprachen als kodierten Problemen zurück.
Viele Einführungen in die theoretische Informatik beginnen 
mit einer speziellen Klasse von formalen Sprachen,
den \emph{regulären Sprachen} und
den Deterministisch Finiten Automaten (DFA),
welche die kodierten Probleme lösen (``berechenen'')
beziehungsweise die entsprechende Sprache erkennen.
In diesem Skript werden diese DFAs 
als eine Art ``Hello World'' der Automatentheorie diskutiert:
Wie beim Erlernen einer neuen Programmiersprache
ist es auch für die Automatentheorie zielführend
ein möglichst einfaches formales Konzept einzuführen,
um daran das generelle ``Handling'' von Sprachen und formalen Werkzeugen zu zeigen. 
DFAs erlauben es die im vorigen Kapitel eingeführten
Konzepte von Speicherplatz und Laufzeit präzise zu fassen.
In einem Exkurs über reguläre Ausdrücke,
wird ein in der Praxis sehr relevantes Handwerkzeug einer IT-Fachkraft vermittelt.
Die ``Berechnungen'', die mit einem DFA möglich sind, 
haben allerdings Grenzen, deren Diskussion das Kapitel abschließt.

Mit den Turing-Maschinen lernen wir in Kapitel 4 einen Formalismus kennen,
der diese Beschränkungen nicht teilt.
Turing-Maschinen sind der zentrale Formalismus sowohl
in der Komplexitätstheorie,
als auch in der Berechnungstheorie.
Auch sie erlauben (Speicher)platz und Laufzeit präzise zu quantifizieren,
auch sie erkennen eine Klasse von Sprachen respektive Probleme.
Darüberhinaus konnte bisher kein Formalismus gefunden werden,
der ``mächtiger'' als Turingmaschinen ist.
Mit den Turing-Maschinen haben wir das nötige Konzept an der Hand,
um die Hauptfrage des Skriptes in den folgenden beiden Kapiteln zu beantworten.

Zentralen Ergebnisse und einige offenen Fragen der Komplexitätstheorie
werden in Kapitel 5 besprochen.
Diese Zusammenfassung ist ein Teil der Antwort auf die Frage,
warum IT-Fachkräfte mit einem starken Hintergrund in theoretischer Informatik
schneller, effizienter und präziser sind:
Zu wissen zu welcher Komplexitätsklasse ein Problem gehört,
macht die Wahl der Mittel, d.h. der nötigen algorithmischen Verfahren zielgerichtet.

Die Theorie der (Un)berechenbarkeit ist das Thema in Kapitel 6.
Wo die Grenzen der automatisierten Informationsverarbeitung liegen,
ist das Wissen,
was Ortskundige in der IT-Landschaft von Zugereisten unterscheidet.
Einige überraschende Ergebnisse der Berechenbarkeitstheorie
bewahren professionelle IT-Kräfte davor,
Zeit und Energie in Fragestellungen zu stecken,
die sich aus prinzipiellen Gründen nicht beantworten lassen.

Kapitel 7 ist ein Exkurs: Kontextfreie Sprachen werden von Kellerautomaten erkannt,
und liegen damit genau zwischen den regulären Sprachen
und denen, die von Turingmaschinen erkannt werden.
Sie spielen im Compilerbau und dem Parsen von Programmiersprachen eine zentrale Rolle.

Kapitel 8 bietet einen Überblick,
wie Interessierte von den dem Inhalt dieses Skriptes tiefer in die theoretische Informatik
eintauchen können.

Die Kapitel bauen jeweils aufeinander auf,
weswegen diese Einführung linear zu lesen ist.
Die meisten Einführungen erlauben Auslassungen und Sprünge,
um nur eine Auswahl der Themen zu studieren.
Dieses Skript ist bereits eine solche Auswahl
und daher ist die Reihenfolge fix
und der Umfang minimal für das gewählte Narrativ.

Wenn wir zurück zur Daseinsberechtigung dieses Skriptes kommen,
stellen wir fest:
Viele zentrale Ergebnisse, Hintergründe und Konzepte
der theoretischen Informatik,
die für den gerade dargestellten Erzählstrang irrelvant sind,
finden sich nur in Fußnoten und Referenzen:
Wir werden nichts über Abzählbarkeit,
wenig über Abschlusseigenschaften formaler Sprachen,
linear-beschränkte Turingmaschinen,
oder Transduktoren zu sagen haben.
Das Ziel des Skriptes ist aber nicht Vollständigkeit in der Breite,
sondern argumentative Kraft in der Tiefe:
Nach der Lektüre wird hoffentlich klar,
warum eine weitere Vertiefung des Faches lohnenswert ist,
auch in Bezug auf diese ausgesparten Konzepte.

Ein zweites Alleinstellungsmerkmal findet sich im Fokus auf grundlegende
wissenschaftsphilosophische und metatheoretische Fragstellungen,
die ansonsten in anderen Einführungen eher am Rande gestreift
oder einfach ignoriert werden.
Was genau meinen wir mit Begriffen wie ``Problem'', ``Berechnung'', ``Effizienz''?
Welche Rolle spielen Konventionen in der theoretischen Informatik?
Welche Alternativen gibt es zum Methodenkoffer der theoretischen Informatik
bei praktischen Fragestellungen
und warum haben letztere trotzdem eine Daseinsberechtigung?
Diese Fragen können leicht abgetan werden, frei nach dem Fregeschen Dictum
``philosophia sunt, non leguntur!''\footnote{
    In etwa: ``Das ist Philosophie,
    das lesen wir (als Informatiker:innen) nicht!''
    Frege beklagt in \cite{frege_grundgesetze} (7),
    dass der Mehrwert seiner Beiträge weder in der akademischen Philosophie,
    noch in der akademischen Mathematik Anerkennung finden.
}
Das Buch verfolgt aber den Ansatz,
dass sich Studierende der theoretischen Informatik
\emph {leichter} mit dem klassischen Handwerkszeug tun,
wenn sie Antworten auf diese Fragestellungen mitdiskutieren.
Die kritische Perspektive der (formal-analytischen) Philosophie

Ein dritter Aspekt, der dieses Skript von anderen Einführungen unterscheidet,
ist der Fokus auf die Komplexitätstheorie,
also die Theorie von Problemklassen,
die sich wesentlich aufgrund ihrer ``Schwierigkeit'' unterscheiden.
Dieser Aspekt und der Zusammhang zwischen theoretischer Informatik und ``Algorithmik''\footnote{
    Wir verstehen darunter das Handwerkzeug, für gegebene Probleme,
    die effizientesten Algorithmen und Datenstrukturen zu identifizieren 
    und zu implementieren}
ist in den meisten Einführungen zur theoretischen Informatik
erst spät im Text oder überhaupt nicht zu finden.
In vielen Vorlesungen wird die Komplexitätstheorie nicht ausreichend diskutiert, 
da im fortgeschrittenen Semester Abstriche gemacht werden müssen.
In diese Situation zu kommen, wird von diesem Skript aufgrund des Aufbaus deutlich erschwert.

Für wen ist dieses Skript nun geschrieben?
Das ursprüngliche Skript wurde begleitend zur Veranstaltung
\emph{Theoretische Informatik (Wirtschaftsinformatik)}
an der DHBW Heilbronn erstellt.
Es richtet sich vor allem an Fachhochschul- und BA-Student:innen,
die Informatik eher als Teil- oder Nebenfach belegen.
Desweiteren wird das Skript aus oben genannten Gründen auch für Studierende der Philosophie
in der analytisch-angelsächischen Tradition (d.h. mit einem Fokus auf formales Arbeiten)
eine gewinnbringende Einführung in das Fach bieten.
Für Studierende der Informatik auf Universitätsniveau kann das Skript 
\emph{ergänzend} hilfreich sein,
sollte allerdings mit einem vollständigen Kompendium ergänzt werden,
um Prüfungen erfolgreich zu bestehen.
Die folgende Aufstellung von Einführungen in die theoretische Informatik
listet geeignete Kandidaten für eine Komplementierung auf
und spiegelt dabei den Geschmack beziehungsweis den Kenntnisstand des Autors wieder
und hat daher keinen Anspruch auf Vollständigkeit:
\begin{itemize}
    \item  \cite{hoffmann} bietet ein sehr umfassendes Kompendium
            und ist für alle geeignet, die sich von den weiteren Alternativen nicht angesprochen fühlen.
    \item  Einen kurzen und konzisen Überblick bietet \cite{schoening}.
    \item  Für eine Compiler-zentrierte Einführung ist \cite{hedtstueck}
           zu empfehlen.
    \item  Wer Wert auf vollständige formale Beweise legt,
           wird bei \cite{erkpriese} fündig.
    \item  \cite{neubert} hat die niedrigeste Einstiegshürde (es wird viel und ausführlich erklärt)
           und hat einen starken Fokus auf Algorithmen bzw. Algorithmik.
    \item  Der (englisch-sprachige) Klassiker ist \cite{hopcroftullman},
           für puristische und historisch interessierte Leser:innen zu empfehlen.
    \item  \cite{barak} ist ein Skript in der Entstehung, es ist englisch und offen im Netz zugänglich.
\end{itemize}

Um das Skript lesen und verstehen zu können,
sind Vorkenntnisse notwendig,
die wir in \autoref{sec:grundlagen} kurz streifen.
Der Abschnitt ist als Wiederholung und Kontrolle gedacht,
nicht als allumfassende Einführung in die genannten Konzepte.
Werden beim Durchlesen Lücken bemerkt,
beinhalet das Kapitel ausreichend Literaturempfehlungen,
um diese zu schließen.

 

  \tableofcontents

  \markboth{Contents}{Contents}

  \mainmatter
  \setcounter{page}{1}
  \chapter{Einleitung}
\section{Warum theoretische Informatik?}
Warum lohnt sich die Beschäftigung mit theoretischer Informatik?
Die kurze Antwort: Sie liefert einen wesentlichen Teil des Handwerkszeugs,
um die Potentiale und Grenzen der technischen Informationsverarbeitung zu umreißen.
Die lange Antwort soll dieses Skript geben.

Das Skript wurde begleitend zur Veranstaltung
''Theoretische Informatik (Wirtschaftsinformatik)''
an der DHBW Heilbronn erstellt.
Wir folgen in diesem Skript einem kartesischen Ansatz:
Wir gehen vom maximalen Zweifel aus und setzen die theoretische Informatik
als potentiellen Nichtsnutz auf die Anklagebank.
Schritt für Schritt werden wir die Jury (die Studierenden) von der Unschuld dieser
Disiplin überzeugen.
Wir beginnen mit der Feststellung, dass uns als Informatiker:innen die Nützlichkeit
eines Werkzeugs dann einleuchtet, wenn es hilft Probleme zu lösen.
Daher ist der Problembegriff im Mittelpunkt dieses Skripts.

Was also ist ein Problem im Sinne der (theoretischen) Informatik?
Bevor wir diese Frage grundlegend beantworten können,
ist es hilfreich ein paar Probleme kennen zu lernen,
wie sie als Aufgaben im IT-Alltag auftreten können.

\section{Probleme}
Aufgaben, die viele IT-Fachkräfte Tag für Tag bearbeiten,
werden oft als Userstories dokumentiert.
Eine Userstory wird in der agilen Softwareentwicklung genutzt um Anforderungen zu spezfizieren.
Eine Userstory ist typischerweise in diesem Format angegeben:
''Als \textless Rolle\textgreater\ möchte ich \textless Feature\textgreater,
um \textless Ziel\textgreater\ zu erreichen.''
Diese Beispiele sollen uns im Skript begleiten:

\begin{itemize}
    \item \textbf{MAX}: Als Filialleiter:in möchte ich den Artikel mit der höchsten Gewinnmarge
        kennen, um meine Lieferungen zu optimieren.
    \item \textbf{MATCH}: Als Einkaufsmanager:in möchte ich,
        dass nur valide Lieferanten-IDs gespeichert werden
        (beginnend mit einem ``L'', gefolgt von einer Zahl),
        um Fehler in der Datenbank zu vermeiden.
    \item \textbf{HALT}: Als Product Owner des Serverless-Computings möchte ich,
        dass nur terminierende Code-Snippets deployt werden,
        um die verfügbaren Ressourcen effizient zu nutzen.
    \item \textbf{ROUTE}: Als Kommissionierer:in möchte ich,
        dass morgens die optimalen Routen für die LKWs in die 24 Filialen berechnet werden,
        um die Lieferzeit zu minimieren.
    \item \textbf{QUANTUMSEC}: Als Chief Security Officer möchte ich,
        dass die firmen-interne Kommunikation quantensicher verschlüsselt wird,
        um die Vertraulichkeit der Absprachen zu garantieren.
\end{itemize}

Mit den Methoden der theoretischen Informatik ausgerüstet,
kann man einiges über diese Alltags-Probleme sagen:
\begin{itemize}
    \item Mindestens ein Problem ist sehr schwer zu lösen.
    \item Mindestens ein Problem ist sehr einfach zu lösen.
    \item Mindestens ein Problem ist nicht lösbar.
    \item Bei mindestens einem Problem ist unklar, ob es sich lösen lässt.
\end{itemize}

Am Ende dieses Skriptes soll klar sein, welches dieser Probleme in welche dieser Kategorien fällt.
An diesem Punkt im Text wollen wir uns aber zunächst damit begnügen,
dass ein Fach Nutzen bringt, dass es erlaubt, diese Probleme auf einen Blick als leicht oder
schwer oder unlösbar zu kategorisieren.

Eine Userstory ist jedoch keine geeignete Form,
um die Methoden der theoretischen Informatik anzuwenden.
Was hier noch zu tun ist, soll im nächsten Abschnitt besprochen werden.

\section{Abstraktion}

Um die Userstories aus dem vorigen Absatz für die Bearbeitung mit den Methoden der
theoretischen Informatik vorzubereiten ist noch einiges an Arbeit notwendig.
Ein grundlegendes Instrument der Informatik (nicht nur der Theoretischen) 
ist die \emph{Abstraktion}, also die ausschließliche Betrachtung der Aspekte eines Sachverhalts,
die für einen gegebenen Zweck wesentlich sind.
Mit der Abstraktion kommen wir also von einer sehr speziellen Aufgabenbeschreibung,
wie einer Userstory, zu einem abstrakten Problem.

Es gibt leider keinen einfachen und eindeutigen Weg
duch Abstraktion von einer Userstory zu einem abstrakten Problem zu kommen.
Dafür sind intellektuelle Kreativität und ein gutes Verständnis des gegebenen Zwecks notwendig.
Ein möglicher Fehler ist z.B. zu wenige Aspekte auszulassen,
so dass das abstrakte Problem nicht so einfach wie möglich ist. 
Ebenso ist es ein Fehler, zu viele Aspekte auszulassen,
so dass das abstrakte Problem nicht so detailreich wie nötig ist,
um dem gegebenen Zweck gerecht zu werden.

Die Behandlung von Problemen mit den Methoden der theoretischen Informatik
ist der Zweck, den wir verfolgen. 
Diesen muss man von den Zielen unterscheiden,
derentwegen die Userstories formuliert wurden.
Die Zwecke können sich überschneiden, sind aber voneinander zu unterscheiden.\\

\noindent
So könnte eine Abstraktion für das \textbf{MAX}-Problem aussehen:
\begin{enumerate}
    \item \emph{Als Filialleiter:in möchte ich den Artikel mit der höchsten Gewinnmarge kennen,
        um meine Lieferungen zu optimieren} (ursprüngliche Userstory).
    \item \emph{Identifikation des Artikels mit der höchsten Gewinnmarge}
        (Wer den besten Artikel aus welchem Grund kennen will,
        ist für unseren Zweck gleichgültig.)
    \item \emph{Identifikation des Artikels,
        der aus einer Menge an Artikeln einen maximalen Wert in Form einer natürlichen Zahl hat}
        (Es ist z.B. egal,
        ob es der beste Artikel im Sinne der Verkaufshäufigkeit
        oder der mit der höchsten Gewinnmarge in Cent ist.)
    \item \emph{Identifikation des Maximums in einer Menge natürlicher Zahlen}
        (Es ist egal, ob wir das Maximum in einer Liste von Artikel oder anderer Objekte suchen.)
\end{enumerate}

Wie wir im Laufe des Skriptes sehen werden, ist die letzte Formulierung des Problems die 
wesentliche Vorstufe, um die Methoden der theoretischen Informatik anzuwenden.\footnote{
Es ist wie gesagt möglich,
dass dieses abstrakte Problem nicht das richtige Resultat für einen anderen Zweck ist.
Für die Filialleiter:innen mag es wesentlich sein,
nach welcher Dimension der ``beste'' Artikel bestimmt wird.
Oder die Ausgabe sollte nicht nur die maximale Zahl sein, sondern die IDs der
fünf besten Artikel.
Wie man Abstraktionen im Sinne der Software Entwicklung einsetzen kann,
lehrt die Disziplin des Software Engineerings.}
Wir kommen so auf den Kern des \textbf{MAX}-Problems:
\begin{itemize}
    \item \textbf{Gegeben:} Ein Tupel von natürlichen Zahlen
    \item \textbf{Gesucht:} Die größte Zahl aus dem Tupel
\end{itemize}
Diese Form der Problemangabe (Gegeben/Gesucht) wollen wir \emph{informell} nennen.
Dabei ist für die informelle Angabe wichtig,
sowohl den Typ des (potentiell komplexen) Eingabewerts,
als auch des (potentiell komplexen) Ausgabewerts zu spezifizieren.
Mit der informellen Problemangabe lassen sich bereits drei sehr grundlegende Fragen stellen:
\begin{enumerate}
    \item Wie muss \textbf{MAX} dargestellt sein, damit eine Maschine es lösen kann?\\ 
        In anderen Worten: Wie kann \textbf{MAX} \emph{formalisiert} werden? 
    \item Kann eine Maschine \textbf{MAX} überhaupt lösen?\\
        In anderen Worten: Ist \textbf{MAX} \emph{berechenbar}?  
    \item Wie effizient kann eine Maschine \textbf{MAX} lösen?\\
        In anderen Worten: Wie \emph{komplex} ist \textbf{MAX}?
\end{enumerate}
Diese Fragen können wir natürlich nicht nur für \textbf{MAX} stellen.
Auch die anderen Userstories von oben lassen sich in eine informelle Form bringen,
die diese Art der Fragestellung erlaubt.
Es gibt drei grundlegenden Teildisziplinen der theoretischen Informatik,
die Antworten auf diese Fragen geben können und deren Rahmenbedingungen untersuchen:
\begin{enumerate}
    \item Die \textbf{Theorie der formalen Sprachen und Automaten}
    \item Die \textbf{Berechenbarkeitstheorie}
    \item Die \textbf{Komplexitätstheorie}
\end{enumerate}

Der Rest des Skriptes wird in die grundlegenden Ergebnisse dieser drei Teildisziplinen einführen,
ihre Wechselwirkung beschreiben
und die Anwendung ihrer Methoden auf die eingangs erwähnten Userstories erläutern.
Dabei werden wir natürlich nicht nur \textbf{MAX} behandeln, sondern uns generell fragen,
welche Probleme formalisiert, überhaupt berechnet und effizient berechnet werden können.
Zunächst wollen wir uns aber noch eine beispielhafte Formalisierung von \textbf{MAX}
anschauen, um dabei gleich weitere grundlegende Begriffe der theoretischen Informatik
einzuführen.

Wir halten fest, dass die Abstraktion ein Werkzeug ist,
Probleme auf ihren Kern zu reduzieren.
Diese abstrakte, informelle Form des Problems hat genau den Umfang,
der so klein wie möglich und so groß wie nötig ist.

\section{Formalisierung}
\noindent
Die informelle Angabe des \textbf{MAX}-Problems legt nahe, dass wir nach einer Möglichkeit suchen
eine Funktion zu berechnen:
\begin{itemize}
    \item Der Definitionsbereich der Funktion ist die Menge aller Tupel von natürlichen Zahlen,
        die wir mit $\cal{N}$ bezeichnen wollen.
        Wir nutzen das Zeichen $\nu$,
        um einen beliebigen Tupel $\nu \in \cal{N}$ zu bezeichnen.
    \item Der Wertebereich der Funktion ist die Menge der natürlichen Zahlen, $\mathbb{N}$.
        Wir nutzen das Zeichen $n$,
        um eine beliebige natürliche Zahl $n \in \mathbb{N}$ zu bezeichnen.
\end{itemize}
Gesucht ist also $f_{MAX}: \cal{N} \rightarrow \mathbb{N}$.
Es gilt zum Beispiel:
\begin{itemize}
    \item Sei $\nu = [2,3]$ und $n = 3$,
        dann gilt: $[[2,3],3] \in f_{MAX}$.
    \item Sei $\nu = [2,3]$ und $n = 2$,
        dann gilt: $[[2,3],2] \notin f_{MAX}$.\footnote{
            TODO: Verweis auf \ref{relationenUndFunktionen} um zu erklären,
            dass Funktionen eine Menge von rechtseindeutigen Tupeln sind.}
\end{itemize}

\noindent
Wie finden wir einen Mechanismus oder eine Maschine, die $f_{MAX}$ berechnet?

Die einfachste Lösung wäre eine große Tabelle, die uns für jedes $\nu$
das jeweilige Maximum $n$ ablesen lässt.
Das wäre zwar nicht sonderlich Zeit-effizient,
da wir viele Vergleiche anstellen müssten,
ob unser gegebenes $\nu$ mit einem Tabelleneintrag übereinstimmt,
aber immerhin wäre die Methode Zeit-effektiv, 
d.h. nach endlich vielen Schritten hätten wir ein Ergebnis.
Allerdings ist $\cal{N}$ so groß,
dass die angenommene Anzahl der Atome in unserem Universum nicht ausreicht,
um alle Einträge der Tabelle zu verwirklichen.
Die Methode ist also auch nicht Platz-effizient und nur dann Platz-effektiv,
wenn wir unendlich viel Platz zur Verfügung haben.

Ein weiterer Ansatz $f_{MAX}$ zu berechnen,
wäre eine Maschine zu erfinden, der wir $\nu$ übergeben,
und die uns $n$ zurückgibt.
Es wäre demnach zu zeigen, ob die Maschine die Aufgabe in endlicher Zeit
und mit endlichen Platz-Ressourcen bewältigen kann.
%Wir könnten die Maschine von der Bedienung her noch ein Stück einfacher gestalten
%und ihr direkt $\nu$ und $n$ übergeben und eine Lampe oder ein Schalter 
%(oder eine andere binäre Ausgabe) zeigt an, ob $[\nu, n] \in f_{MAX}$.
Nehmen wir in einem Gedankenexperiment an, wir hätten eine solche Maschine erfunden.
Wie müssten wir den Input und Output der Maschine gestalten,
d.h. in welche Form müssten wir das \textbf{MAX}-Problem bringen,
dass die Maschine es lesen kann und wir den Output wieder verstehen können? 

Ein zugängliches Modell für den Input ist die Tastatur und der Bildschirm:
Eine endliche Menge von Zeichen (angezeigt durch Drücken der Knöpfe einer Tastatur)
werden von der Maschine übersetzt in eine Reaktion der Maschine 
(angezeigt durch die Pixel einer digitalen Ausgabe).
Tatsächlich werden die meisten Informationen,
die eine Maschine oder ein Computer verarbeiten,
intern in eine sehr kleine Menge an Zeichen übersetzt:
in Zeichenfolgen bestehend aus Nullen und Einsen.
Tastaturen und Bildschirme machen uns die Interaktion mit der Maschine einfacher,
allerdings sind sie für die Maschine selbst nicht nötig
(unsere Tastatur könnte also nur drei Knöpfe beinhalten: 0, 1 und die Enter-Taste,
die Ausgabe könnte ein Drucker sein, der 0 und 1 auf ein Papierband schreibt).

%IT-Systeme kodieren Zahlen, Videos, Bilder, Musik, Texte, im Allgemeinen alle Datenstrukturen
%und Ausgabewerte als Folgen von Nullen und Einsen,
%also als binäre Zeichenfolgen.
%Ebenso werden Anweisungen, was mit diesen binären Daten passieren soll,
%als binäre Zeichenfolgen kodiert.

Ganz allgemein könnte man also sagen, dass alle Probleme, die ein Computer lösen soll,
sich in diese Form bringen lassen:
\begin{itemize}
    \item \textbf{Gegeben:} Eine binäre Zeichenfolge.
    \item \textbf{Gesucht:} Eine binäre Zeichenfolge.
\end{itemize}
%Oder eben gegeben als Funktion $f: \mathcal{P}(\{0,1\}) \rightarrow \{0,1\}$
Übertragen auf \textbf{MAX} müssen wir eine binäre Repräsentation von $\nu$ und $n$ finden.
Damit wir dies auf eine methodische Art und Weise erreichen
und die gefundenen Methoden später von \textbf{MAX} auf andere Probleme übertragen können,
müssen wir zunächst noch einige Begriffe definieren und Konventionen einführen.

\subsection{Zeichen}

Eine Menge von Zeichen bezeichnen wir als \emph{Alphabet}, kurz: $\Sigma$.
Zum Beispiel wäre das Alphabet für die meisten IT-Systeme: $\Sigma = \{0,1\}$.\\

\noindent
Wir verwenden diese Variablen für Zeichen: $a,b,c$;
wenn wir diese Buchstaben verwenden und nichts anderes angeben,
gilt implizit: $a, b, c \in \Sigma$ für ein gegebenes $\Sigma$.\footnote{
Wenn $\Sigma$ nicht explizit angegeben oder aus dem Kontext herleitbar ist,
wird $\Sigma$ also auch als Variable für (irgend) ein Alphabet verwendet.}\\

\noindent
Eine Operation auf den Zeichen ist die \emph{Konkatenation},
also die Verbindung von zwei Zeichen, z.B. $01$.
Wie bei der Multiplikation mit Variablen in der Mathematik üblich,
wird die Konketanation nicht explizit mit einem eigenen Zeichen repräsentiert.

\subsection{Worte}
Das Ergebnis einer beliebig oft ausgeführten Konkatenation,
also eine (binäre) Zeichenfolge,
bezeichnen wir als \emph{Wort}, z.B.: $0101010001001001$\\

\noindent
Die Menge aller Wörter,
die sich aus einem Alphabet bilden lassen,
bezeichnen wir als \emph{Kleensche Hülle}, dargestellt als Stern: $\Sigma^*$.\\

\noindent
Wir verwenden diese Variablen für Worte:
$u, v, w, x, y, z$; wenn wir diese Buchstaben verwenden und nichts anderes angeben,
gilt implizit: $u, v, w, x, y, z \in \Sigma^*$.\\

\noindent
Die \emph{Länge} eines Wortes ist die Anzahl seiner Zeichen.
Formal benutzen wir die Betragsstriche um ein Wort, um seine
Länge zu bezeichnen. Wenn z.B. $w = 0101010001001001$, dann gilt $|w| = 16$.\\

\noindent
Das Wort mit der Länge 0, also das Wort, dass aus null Zeichen besteht,
nennen wir das \emph{leere Wort}, kurz: $\epsilon$.
Es gilt $|\epsilon| = 0$.\\

\noindent
Für die n-malige Wiederholung eines Zeichens oder Wortes führen wir den \emph{Wiederholungsoperator}
ein, eine hochgestellte Zahl $n \in \mathbb{N}$: $a^n$ oder $w^n$.
Z.B. $0^3 = 000$ und $(10)^2 = 1010$.
$a^0$ bzw. $w^0$ entspricht dabei $\epsilon$.\\

\noindent
Wollen wir ein angeben, dass ein Zeichen oder Wort beliebig oft (also auch 0-mal) wiederholt
wird, nutzen wir dafür einen hochgestellten Stern, den \emph{Kleene-Operator} $a^*$ oder $w^*$.
Mit dem Kleene-Operator bezeichnen wir immer eine unendliche Menge von Wörtern, z.B.:
$a^* = \{\epsilon, a, aa, aaa, aaaa, \dots\}$

\subsection{Formale Sprachen}

Nachdem wir von Zeichen und Worten gesprochen haben, liegt die Frage nahe,
ob man denn auch von Sprachen sprechen kann und tatsächlich ist die formale Sprache 
ein zentraler Begriff der theoretischen Informatik.
Bevor wir den Begriff der formalen Sprachen einführen, ist noch eine Warnung angebracht:
Während die Verwendung des Begriffs ''Zeichens''
noch recht intuitiv mit dem Begriff des Buchstaben zusammenfällt
(zumindest im deutschsprachigen Raum),
ist schon der formal eingeführte Wort-Begriff eher eine Zumutung für unsere Intuition:
Worte im Sinne des vorangegangenen Abschnitts sind einfach irgendwelche Zeichenfolgen,
egal ob sie aussprechbar oder im Kontext einer natürlichen Sprache
(wie deutsch oder englisch) einen Sinn ergeben.

Man müsste also eigentlich das natursprachliche $Wort_N$ von dem formalen $Wort_F$ trennen,
das wir oben eingeführt haben.
Eine Crux aller formal arbeitenden Wissenschaften ist,
dass die Menge an Zeichen und Begriffen begrenzt ist und daher Doppeldeutigkeiten auftreten.
Wir müssen also durch Konventionen klären, welche Bedeutung wir einem Begriff geben wollen:
Im Kontext dieses Skriptes bezeichnen wir mit dem Begriff Wort die Zeichenfolge, also $Wort_F$.
Sollte dies nicht der Fall sein und der Kontext nicht ausreichen,
um zwischen beiden Bedeutungen zu unterscheiden,
werden wir die subskribierten Varianten verwenden.

Die vorige Überlegung trifft nicht nur auf Worte, sondern auch auf \emph{Sprachen} zu.
Eine Sprache im Sinne der theoretischen Informatik ist einfach eine Menge von Worten.
Z.B. ist $L = \{0, 1, 00, 01, 10, 11 \}$ die Sprache der Worte,
die aus einem oder zwei Binärzeichen bestehen.
Wesentlich ist eine formale Sprache als \emph{Menge} im Sinne der Mengenlehre aufzufassen, 
nicht als Kommunikationsmedium mit einer eigenen Sprachgeschichte und Sprachgemeinschaft.\\

\noindent
Wir verwenden $L$ (ggf. mit unterschiedlichen Subskripten) als Variable für Sprachen;
wenn wir $L$ verwenden und nichts Anderslautendes angeben ist,
gilt implizit $L \subseteq \Sigma^*$ für ein gegebenes $\Sigma$.\\

\noindent
Da Sprachen Mengen sind,
sind alle mengentheoretischen Operationen\footnote{siehe \autoref{mengenlehre}}
auf Sprachen definiert,
z.B. bezeichnet $L_1 \cap L_2 = \{w | w \in L_1 \wedge w \in L_2\}$
die Schnittmenge zweier Sprachen, also alle Wörter,
die sowohl Element von $L_1$ als auch von $L_2$ sind.

\subsection{Probleme als Sprachen}
Mit den oben eingeführten Begriffen können
wir nun ein Problem wie \textbf{MAX} im Sinne der theoretischen Informatik
\emph{formalisiert} spezifizieren.
Die grundlegende Idee dabei: Ein Problem lässt sich durch eine formale Sprache repräsentieren. 
Die formale Sprache besteht dabei aus der binären Repräsentation genau der Paare $\nu$, $n$,
für die gilt $[\nu, n] \in f_{MAX}$. 
Die Sprache ist also der großen Tabelle nicht unähnlich,
in der man die ``richtigen'' $n$-Werte ablesen kann.
Wir wissen, dass diese Sprache unendlich viele Wörter hat und dass es unwirtschaftlich wäre
die Sprache wirklich aufzuschreiben (wir würden nie fertig werden).
Perspektivisch müssen wir also nach endlichen Methoden suchen,
(potentiell) unendliche Sprachen zu verarbeiten.

Wie genau können wir aber ein Problem wie \textbf{MAX}
als Menge binärer Zeichenfolgen darstellen?
Bisher haben wir ja mit $\cal{N}$ und $\mathbb{N}$ mehr als zwei Zeichen genutzt:
also $\Sigma = \{0,1,2,3,4,5,6,7,8,9,[,],``,''\}$
Dafür müssen wir zunächst eine Binärkodierung für Zahlen finden:
$Bin_{LE}: \mathbb{N} \rightarrow \{0,1\}$
Die Little Endian Dual-Kodierung erfüllt diese Anforderung,
also die Repräsentation von natürlichen Zahlen bei der
das niedrigstwertige Bit am Anfang steht.
Zum Beispiel ist $Bin_{LE}(6) = 011$, mit 
$
0 \cdot 2^0
+ 1 \cdot 2^1
+ 1 \cdot 2^2
= 6
$.\footnote{TODO: Füge Ressource hinzu, die Dual-Kodierung und LE/BE erklärt;
entweder als Referenz oder in Kapitel 1}

Mit $Bin_{LE}$ können wir $\Sigma$ auf die folgende Menge reduzieren: $\{0, 1, [, ], ``,''\}$.
Damit wir nun nicht nur einfache Zahlen, sondern auch Zahlenfolgen kodieren können,
müssen wir noch eine weitere Kodierung einführen,
die es erlaubt an einer binärem Wort zu erkennen,
wo eine Zahl anfängt und die nächste beginnt.
Dies kann mit der folgenden Kodierung erreicht werden: 
$Bin_{TUPLE}:$ 

\begin{itemize}
    \item $Bin_{TUPLE}(0) = 00$
    \item $Bin_{TUPLE}(1) = 01$
    \item $Bin_{TUPLE}(,) = 11$
\end{itemize}
Z.B. ist $Bin_{TUPLE}(11,10,01,00) = 0101$ $11$ $0100$ $11$ $0001$ $11$ $0000$.
Damit haben wir $\Sigma$ auf vier Zeichen reduziert: $\{0,1,[,]\}$.
Wir könnten nun eine geeignete binäre Repräsentation von Tupeln finden,
um beliebige Tiefen in der Schachtelung zu realisieren.
Allerdings gibt es eine einfachere Möglichkeit: Konvention.
Wir legen fest, dass die letzte binär kodierte Zahl $n$ entspricht,
alle anderen binär kodierten Zahlen gehören zu $\nu$.

Für die Sprache $L_{MAX}$, die das Problem \textbf{MAX} kodiert oder auch repräsentiert,
können wir also folgende Aussagen treffen:
\begin{itemize}
    \item $0100$ $11$ $0101$ $11$ $0101$ $\in L_{MAX}$, da $f_{MAX}([[2,3]) = 3$
    \item $0100$ $11$ $0101$ $11$ $0100$ $\notin L_{MAX}$, da $f_{MAX}([[2,3]) \neq 2$
\end{itemize}

Sind nun Probleme identisch mit den formalen Sprachen, die sie repräsentieren? 
Die Antwort auf diese Frage hat sehr viele philosophische Implikationen,
die wir nicht im Rahmen dieses Skriptes behandeln können.\footnote{Sprachphilosophisch
könnten wir uns fragen, ob die Übersetzung eines Problems in eine formale Sprache
überhaupt eindeutig und verlustfrei möglich ist.
Oder ob es überhaupt so etwas wie Probleme gibt, die von ihrer repräsentierten Sprache 
getrennt existieren.}
Fakt ist zum einen, dass das \emph{Wortproblem}, also ob für ein Wort $w$ und eine Sprache $L$
gilt $w \in L$,
eng verknüpft ist mit einem abstrakten Problem, das von der formalen Sprache repräsentiert wird. 
Fakt ist ebenfalls, dass in vielen Textbüchern die formale Sprache und ihr Problem
im Prinzip miteinander austauschbare Entitäten sind. 
Wir wollen allerdings sorgsam sein und die beiden voneiander trennen.\\
%Wir wollen in unserem Skript allerdings davon sprechen, dass eine Sprache ein Problem
%repräsentiert, ohne eine Aussage zu treffen,
%ob die beiden identisch oder voneinander zu unterscheiden sind,
%da wir uns vorgenommen haben das Potential der theoretischen Informatik für das Lösen
%von IT-Problemen zu bewerten.
%Die spannenden philosophische Diskussionen lassen wir hier aus. 

\noindent
Um von unserer Userstory zu $L_{MAX}$ zu gelangen,
haben wir einen beträchtlichen formalen Aufwand investiert.
Daher stellt sich die Frage: Ist dieser Aufwand gerechtfertigt? 
Warum also geben wir Probleme als Wortprobleme von formalen Sprachen an?
Dies sind die wesentlichen Antworten auf beide Fragen,
nach Wichtigkeit aufsteigend sortiert:
\begin{itemize}
    \item Abstrakte Probleme mit Sprachen zu identifizieren gibt uns eine
        konzise, unmissverständliche Beschreibung von Problemen,
        die auch von Maschinen verarbeitet werden kann. 
    \item Die Abstraktion die in der Formalisierung wiedergespiegelt ist,
        ermöglicht sehr generelle Aussagekraft über das speziell betrachtete Problem hinweg.
    \item Aussagen über (effiziente) Berechenbarkeit werden mit den Methoden der Mathematik
        beweisbar, da wir u.a. die Ergebnisse der Mengenlehre, der Arithmetik
        und der Analysis anwenden können.
\end{itemize}

Zusammenfassend können wir feststellen:
\begin{itemize}
    \item Die theoretische Informatik arbeitet mit Abstraktion und Formalisierung.
    \item Probleme lassen sich als Sprachen repräsentieren.
        Ist es lösbar, ob ein Wort zu einer Sprache gehört,
        ist auch das repräsentierte Problem lösbar. 
    \item Die abstrakten Probleme der theoretischen Informatik lassen sich mit
        mathematischen Werkzeugen bearbeiten,
        wodurch ihre Ergebnisse theoretisch gefestigt sind 
        und die Aussagekraft sehr generell ist.
\end{itemize}
 
  aufwand.tex 
  aufwand.tex 
  aufwand.tex
  \chapter{Der Barbier und der Lügner}\label{derBarbierUndDerLuegner}
\section{Paradoxien}
\section{LIAR \& HALT} 
\section{Berechnungstheoretische Reduktionen}
\section{Der Satz von Rice}
\section{Formale Verifikation und Spezfikation}

  komplexitaet.tex
  \chapter{Absprung in die reine Theorie}
\section{Chomsky und Grammatiken}
\section{Compiler und Kellerautomaten}
\section{Weitere Komplexitätsklassen}
\section{Automaten und Logik}

  theorie.tex

  \backmatter
  \appendix
  \renewcommand{\thesection}{\Alph{section}}
  \setcounter{section}{0}
  \setcounter{subsection}{0}
  \setcounter{table}{0}
  \renewcommand{\thetable}{A.\arabic{table}}
  \titleformat*{\subsection}{\normalsize\bfseries}
  \chapter{Appendix}
  \chapter{Logisch-Mathematische Prolegomena}
Dieses Kapitel soll in konziser Weise die Ergebnisse der folgenden Fächer zusammenfassen,
die Voraussetzung für ein Verständnis des Methodenkoffer der theoretischen Informatik sind:
\begin{itemize}
    \item Formale Logik (\autoref{al}, \autoref{pl}, \autoref{bew})
    \item Mengenlehre (\autoref{mengenlehre} und \autoref{relationenUndFunktionen})
\end{itemize}
Jedes dieser Kapitel ist knapp gehalten und hat einen Anreißer,
der es den Lesenden erlaubt schnell zu entscheiden,
ob eine Auffrischung vonnöten ist.
Die Abschnitte sind keine vollwertige Einführung in das jeweilige Gebiet,
hierfür sei v.a. \cite{link} empfohlen.

\section{Aussagenlogik}\label{al}
\section{Prädikatenlogik}\label{pl}
\section{Mengenlehre}\label{mengenlehre}
\begin{itemize}
    \item Mengen
    \item Mengendefinition: intensional und extensional
    \item Mengenalgebra
    \item Potenzmenge
    \item Universum
\end{itemize}
\section{Relationen und Funktionen}\label{relationenUndFunktionen}
In diesem Abschnitt werden wir die folgenden Sachverhalte erläutern
(in der umgekehrten Reihenfolge):
\begin{itemize}
    \item Funktionen sind rechtseindeutige Relationen.
    \item Relationen sind Mengen von Tupeln mit identischer Länge.
    \item Tupel sind Mengen mit Reihenfolge.
\end{itemize}
Dies erlaubt uns Funktionen als Menge von Tupeln zu begreifen,
was uns für endliche Funktionen
(wie z.B. eine Übergangsfunktion eines Automaten)
die Funktion extensiv anzugeben (siehe voriger Abschnitt).

\subsection{Tupel}

Für Mengen gilt allgemein:
\[ \{a,b\} = \{b,a\} = \{a,b,a\} \]
Das heißt:
\begin{itemize}
    \item Die Elemente einer Menge haben keine Reihenfolge.
    \item Duplikate eines Elements in einer Extension werden ignoriert.
\end{itemize}
Ein Tupel ist wie eine Menge, nur gilt hier:
\[[a,b] \neq [b,a] \text{ und } [a,b] \neq [a,b,a] \text{ und } [b,a] \neq [a,b,a]\]
D.h. Tupel werden anstatt mit geschweiften mit eckigen Klammern angegeben,
haben eine Reihenfolge und daher können Duplikate nicht einfach ignoriert werden.

\noindent
Das Folgende sollten bekannt sein:
\begin{itemize}
    \item Ein Tupel mit zwei Elementen heißt geordnetes Paar.
    \item Ein Tupel mit drei Elementen heißt Tripel. 
    \item Ein Tupel mit n    Elementen heißt n-Tupel.
    \item Das kartesische Produkt zweier Mengen A und B wird mit $A \times B$ bezeichnet
          und ist definiert als: $\{[a,b]|a \in A \wedge b \in B\}$.
    \item Analog lässts sich das kartesische Produkt von mehr als zwei Mengen definieren.
\end{itemize}

\subsection{Relationen}

Eine Relation ist eine Menge von gleichlangen Tupeln.
Relationen (wie auch Funktionen) bringen eine gewisse Struktur (manchmal auch eine Ordnung)
in ein Universum.
Relationen kann man im Sinne von \autoref{pl} auch als mehrstelliges Prädikat auffassen.
Sei z.B.
\[
    \Sigma = \{a,b,c,...,z\}
\]
unser Universum, also das deutschsprachige Alphabet,
bestehend aus 26 kleinen Buchstaben.\footnote{
Ohne Beschränkung der Allgemeinheit ignorieren wir hier Umlaute und Großbuchstaben.}
Eine mögliche Relation ist die Folgende:
\[
    ABC = \{[a,b], [b,c], [c,d], \ldots, [x,y], [y,z]\}
\]
$ABC$ ist die Relation eines Buchstabens zu seinem Nachfolger
gemäß der Konvention in deutschsprachigen Ländern.
$ABC$ ist ein zweistelliges Prädikat, daher sind alle Elemente der Relation geordnete Paare.
Man kann das bestehen einer Relation zwischen zwei Elementen des Universums
auch in der Infix-Notation angeben, z.B.: aABCb (a steht in der ABC-Relation zu b).

Ein weiteres Beispiel ist:
\[
    \begin{array}{lllllll}
        <_{ABC} &
        = &
            \{[a,b], &
            [a,c], &
            \ldots, &
            [a,y], &
            [a,z],\\ 
        &
        & 
        &
            [b,c], &
            \ldots, &
            [b,y], &
            [b,z], \\
        &
        &
        &
        &
            \ldots,\\
        &
        &
        &
        &
        &     
        &
            [y,z] \}
\end{array}
\]
$<_{ABC}$ gibt die gesamte Ordnung der Buchstaben an (nicht nur den unmittelbaren Nachfolger).
Analog zu oben gibt es auch hier eine Infix-Notation,
die den mathematischen Sehgewohnheiten entspricht: $a <_{ABC} d$ oder $f <_{ABC} x$.

\noindent
Folgende Sachverhalte sollten bekannt sein:
\begin{itemize}
    \item Eine Relation $R$ ist symmetrisch, wenn gilt: $\forall x,y(xRy \leftrightarrow yRx)$.
    \item Eine Relation $R$ ist reflexiv, wenn gilt: $\forall x(xRx)$.
    \item Eine Relation $R$ ist transitiv,
        wenn gilt: $\forall x,y,z(xRy \wedge yRz \rightarrow xRz)$.
    \item Eine Relation, die symmetrisch, reflexiv und transitiv ist,
          heißt Äquivalenzrelation (z.B. ist $=$ eine Äquivalenzrelation).
\end{itemize}

\subsection{Funktionen}
Funktionen sind rechtseindeutige Relationen,
d.h. für den Fall einer zweistelligen Relation R:
\[
    \forall x,y (xRy \rightarrow \neg \exists z (z \neq y \wedge xRz))
\]
Intuitiv bedeutet das: Wenn x der Input der Funktion ist,
dann ist der Output y der Funktion eindeutig bestimmt
(es gibt keinen anderen Wert z der von y unterschieden ist,
der auch Output sein kann, wenn x gegeben ist). 
Da der Output meistens rechts steht, sagt man auch die Relation ist \emph{rechts}eindeutig. 
Die Relation $ABC$ aus dem vorangegangenen Abschnitt ist eine Funktion,
die Relation $<_{ABC}$ nicht.

Folgende Sachverhalte sollten bekannt sein:
\begin{itemize}
    \item Eine andere Bezeichnung für Funktion ist ``Abbildung''.
    \item ''Input'' einer Funktion wird auch Definitionsmenge genannt.
    \item ''Output'' einer Funktion wird auch Wertemenge genannt.
    \item Bei einstelligen Funktionen (also rechtseindeutigen, zweistelligen Relationen)
        ist die Definitionsmenge die Menge aller Elemente,
        die an der ersten Stelle des geordneten Paares stehen.
    \item Bei einstelligen Funktionen (also rechtseindeutigen, zweistelligen Relationen)
        ist die Wertemenge die Menge aller Elemente,
        die an der zweiten Stelle des geordneten Paares stehen.
    \item Bei mehrstelligen Funktionen muss eine Signatur angeben, was zur Definitions-
        bzw. Wertemenge gehört. Z.B. hätte die Funktion $between$,
        die sowohl Vorgänger als auch Nachfolger im Alphabet angibt die folgende Signatur:
        \[ between: 
            \Sigma \backslash \{a,z\}
            \rightarrow
            \Sigma \backslash \{z\} \times \Sigma \backslash \{a\} \]
\end{itemize}

\section{Beweisstrategien \& Beweisskizzen}\label{bew}
 
  \section{Lösungen}
3.3\\
(a) $M = [\Sigma, \Sigma \cup \square, Z, \delta, z_1, \square, E]$ mit
\begin{itemize}
    \item $\Sigma = \{|,-\}$
    \item $Z = \{z_i, z_l, z_g, z_p, z_m, z_d, z_c, z_n, z_r\}$
    \item $
        \begin{array}{llll}
            \delta = \{&
                [|, z_i, z_i, |, R], &
                [-, z_i, z_i, -, R], &
                [\square, z_i, z_l, \square, L], \\
            &
                [|, z_l, z_m, \square, L], &
                [-, z_l, z_g, |, L], \\
            &
                [|, z_g, z_g, |, L], &
                [\square, z_g, z_p, \square, R], \\
            &
                [|, z_m, z_m, |, L], &
                [-, z_m, z_m, -, L], &
                [\square, z_m, z_d, \square, R], \\
            & 
                [|, z_d, z_c, \square, R], \\
            & 
                [|, z_c, z_r, |, R], &
                [-, z_c, z_n, |, R], \\
            &
                [\square, z_n, z_n, -, N], \\
            &
                [|, z_r, z_r, |, R], &
                [-, z_r, z_r, -, R], &
                [\square, z_r, z_l, \square, L] \\
            \}
        \end{array}
    $
    \item $E = \{z_n, z_p\}$
\end{itemize}
(b)
$
\begin{array}{llrlrl}
    [ & [ z_i, &    \square, &                  |, &        |-|] & , \\
      & [ z_i, &    \square|, &                 |, &        -|] & , \\
      & [ z_i, &    \square||, &                -, &        |] & , \\
      & [ z_i, &    \square||-, &               |, &        \square] & , \\
      & [ z_i, &    \square||-|, &              \square, &  \square] & , \\
      & [ z_l, &    \square||-, &               |, &        \square\square] & , \\
      & [ z_m, &    \square||, &                -, &        \square\square\square] & , \\
      & [ z_m, &    \square|, &                 |, &        -\square\square\square] & , \\
      & [ z_m, &    \square, &                  |, &        |-\square\square\square] & , \\
      & [ z_m, &    \square, &                  \square, &  ||-\square\square\square] & , \\
      & [ z_d, &    \square\square, &           |, &        |-\square\square\square] & , \\
      & [ z_c, &    \square\square\square, &    |, &        -\square\square\square] & , \\
      & [ z_r, &    \square\square\square|, &   -, &        \square\square\square] & , \\
      & [ z_r, &    \square\square\square|-, &  \square, &  \square\square] & , \\
      & [ z_l, &    \square\square\square|, &   - &         \square\square\square] & , \\
      & [ z_g, &    \square\square\square, &    | &         |\square\square\square] & , \\
      & [ z_g, &    \square\square, &           \square &   ||\square\square\square] & , \\
      & [ z_p, &    \square\square\square, &    | &         |\square\square\square] & ]\\
\end{array}
$

(c) M berechnet die Subtraktion auf unär kodierten ganzen Zahlen.
5.1
\begin{figure}[ht] % ’ht’ tells LaTeX to place the figure ’here’ or at the top of the page
\centering % centers the figure
\begin{tikzpicture}
    \node[state, initial] (zi) {$\{z_i\}$};
    \node[state, right of=zi] (z12) {$\{z_1,z_2\}$};
    \node[state, right of=z12, accepting] (z23) {$\{z_2,z_3\}$};
    \node[state, below of=zi,  accepting] (z3) {$\{z_3\}$};
    \node[state, right of=z3] (z1) {$\{z_1\}$};
    \node[state, right of=z1, accepting] (z13) {$\{z_1,z_3\}$};
	\node[state, below of=z1] (ze) {$\emptyset$};
	\draw 
    (zi) edge[above] node{0} (z12)
    (zi) edge[left] node{1} (z3)
    (z12) edge[above] node{0} (z23)
    (z12) edge[left] node{1} (z1)
    (z23) edge[loop right] node{0} (z23)
    (z23) edge[left] node{1} (z1)
    (z3) edge[left] node{0,1} (ze)
    (z1) edge[left] node{0} (ze)
    (z1) edge[above] node{1} (z13)
    (z13) edge[above] node{0} (ze)
    (z13) edge[loop right] node{1} (z13)
    ;


\end{tikzpicture}
\caption{Lösungsautomat zu Aufgabe 5.1}
\label{fig:sol51}
\end{figure}
 
  \section{Abbildungsverzeichnis}
  \listoffigures
  \clearpage

  \markboth{Tabellenverzeichnis}{Tabellenverzeichnis}
  \section{Tabellenverzeichnis}
  \listoftables
  \clearpage
   
%  \section{Glossary}
%  \renewcommand{\glossarysection}[2][Glossary]{Glossary}
%  \setglossarysection{section}\label{glossary}
%  \printglossary[numberedsection]{}
%  \clearpage

  \markboth{Bibliographie}{Bibliographie}
  \section{Bibliographie}
  \nocite{*}
  \printbibliography[heading=none]
  \newpage



  \markboth{Danksagung}{Danksagung}
  \section{Danksagung}
  Am Skript haben mitgeholfen
(Korrektur von Rechtschreibfehlern, Fehlerkorrekturen, Formulierungsverbesserungen):
\begin{itemize}
\item Tony Klein (WI21A3)
\item Samis Ellsässer (WI21A3)
\item Katharina Bäuml (WI22A3)
\item Stefan Mack (WI22A3)
\item Jan Hoang (WI22A3)
\item Luan Demaku (WI23A3)
\end{itemize}
Vielen Dank, dass Ihr das Skript besser gemacht habt!


\end{document}
