\chapter{Logisch-Mathematische Prolegomena}
Dieses Kapitel soll in konziser Weise die Ergebnisse der folgenden Fächer zusammenfassen,
die Voraussetzung für ein Verständnis des Methodenkoffer der theoretischen Informatik sind:
\begin{itemize}
    \item Formale Logik (\autoref{al}, \autoref{pl}, \autoref{bew})
    \item Mengenlehre (\autoref{mengenlehre} und \autoref{relationenUndFunktionen})
\end{itemize}
Jedes dieser Kapitel ist knapp gehalten und hat einen Anreißer,
der es den Lesenden erlaubt schnell zu entscheiden,
ob eine Auffrischung vonnöten ist.
Die Abschnitte sind keine vollwertige Einführung in das jeweilige Gebiet,
hierfür sei v.a. \cite{link} empfohlen.

\section{Aussagenlogik}\label{al}
\section{Prädikatenlogik}\label{pl}
\section{Mengenlehre}\label{mengenlehre}
\begin{itemize}
    \item Mengen
    \item Mengendefinition: intensional und extensional
    \item Mengenalgebra
    \item Potenzmenge
    \item Universum
\end{itemize}
\section{Relationen und Funktionen}\label{relationenUndFunktionen}
In diesem Abschnitt werden wir die folgenden Sachverhalte erläutern
(in der umgekehrten Reihenfolge):
\begin{itemize}
    \item Funktionen sind rechtseindeutige Relationen.
    \item Relationen sind Mengen von Tupeln mit identischer Länge.
    \item Tupel sind Mengen mit Reihenfolge.
\end{itemize}
Dies erlaubt uns Funktionen als Menge von Tupeln zu begreifen,
was uns für endliche Funktionen
(wie z.B. eine Übergangsfunktion eines Automaten)
die Funktion extensiv anzugeben (siehe voriger Abschnitt).

\subsection{Tupel}

Für Mengen gilt allgemein:
\[ \{a,b\} = \{b,a\} = \{a,b,a\} \]
Das heißt:
\begin{itemize}
    \item Die Elemente einer Menge haben keine Reihenfolge.
    \item Duplikate eines Elements in einer Extension werden ignoriert.
\end{itemize}
Ein Tupel ist wie eine Menge, nur gilt hier:
\[[a,b] \neq [b,a] \text{ und } [a,b] \neq [a,b,a] \text{ und } [b,a] \neq [a,b,a]\]
D.h. Tupel werden anstatt mit geschweiften mit eckigen Klammern angegeben,
haben eine Reihenfolge und daher können Duplikate nicht einfach ignoriert werden.

\noindent
Das Folgende sollten bekannt sein:
\begin{itemize}
    \item Ein Tupel mit zwei Elementen heißt geordnetes Paar.
    \item Ein Tupel mit drei Elementen heißt Tripel. 
    \item Ein Tupel mit n    Elementen heißt n-Tupel.
    \item Das kartesische Produkt zweier Mengen A und B wird mit $A \times B$ bezeichnet
          und ist definiert als: $\{[a,b]|a \in A \wedge b \in B\}$.
    \item Analog lässts sich das kartesische Produkt von mehr als zwei Mengen definieren.
\end{itemize}

\subsection{Relationen}

Eine Relation ist eine Menge von gleichlangen Tupeln.
Relationen (wie auch Funktionen) bringen eine gewisse Struktur (manchmal auch eine Ordnung)
in ein Universum.
Relationen kann man im Sinne von \autoref{pl} auch als mehrstelliges Prädikat auffassen.
Sei z.B.
\[
    \Sigma = \{a,b,c,...,z\}
\]
unser Universum, also das deutschsprachige Alphabet,
bestehend aus 26 kleinen Buchstaben.\footnote{
Ohne Beschränkung der Allgemeinheit ignorieren wir hier Umlaute und Großbuchstaben.}
Eine mögliche Relation ist die Folgende:
\[
    ABC = \{[a,b], [b,c], [c,d], \ldots, [x,y], [y,z]\}
\]
$ABC$ ist die Relation eines Buchstabens zu seinem Nachfolger
gemäß der Konvention in deutschsprachigen Ländern.
$ABC$ ist ein zweistelliges Prädikat, daher sind alle Elemente der Relation geordnete Paare.
Man kann das bestehen einer Relation zwischen zwei Elementen des Universums
auch in der Infix-Notation angeben, z.B.: aABCb (a steht in der ABC-Relation zu b).

Ein weiteres Beispiel ist:
\[
    \begin{array}{lllllll}
        <_{ABC} &
        = &
            \{[a,b], &
            [a,c], &
            \ldots, &
            [a,y], &
            [a,z],\\ 
        &
        & 
        &
            [b,c], &
            \ldots, &
            [b,y], &
            [b,z], \\
        &
        &
        &
        &
            \ldots,\\
        &
        &
        &
        &
        &     
        &
            [y,z] \}
\end{array}
\]
$<_{ABC}$ gibt die gesamte Ordnung der Buchstaben an (nicht nur den unmittelbaren Nachfolger).
Analog zu oben gibt es auch hier eine Infix-Notation,
die den mathematischen Sehgewohnheiten entspricht: $a <_{ABC} d$ oder $f <_{ABC} x$.

\noindent
Folgende Sachverhalte sollten bekannt sein:
\begin{itemize}
    \item Eine Relation $R$ ist symmetrisch, wenn gilt: $\forall x,y(xRy \leftrightarrow yRx)$.
    \item Eine Relation $R$ ist reflexiv, wenn gilt: $\forall x(xRx)$.
    \item Eine Relation $R$ ist transitiv,
        wenn gilt: $\forall x,y,z(xRy \wedge yRz \rightarrow xRz)$.
    \item Eine Relation, die symmetrisch, reflexiv und transitiv ist,
          heißt Äquivalenzrelation (z.B. ist $=$ eine Äquivalenzrelation).
\end{itemize}

\subsection{Funktionen}
Funktionen sind rechtseindeutige Relationen,
d.h. für den Fall einer zweistelligen Relation R:
\[
    \forall x,y (xRy \rightarrow \neg \exists z (z \neq y \wedge xRz))
\]
Intuitiv bedeutet das: Wenn x der Input der Funktion ist,
dann ist der Output y der Funktion eindeutig bestimmt
(es gibt keinen anderen Wert z der von y unterschieden ist,
der auch Output sein kann, wenn x gegeben ist). 
Da der Output meistens rechts steht, sagt man auch die Relation ist \emph{rechts}eindeutig. 
Die Relation $ABC$ aus dem vorangegangenen Abschnitt ist eine Funktion,
die Relation $<_{ABC}$ nicht.

Folgende Sachverhalte sollten bekannt sein:
\begin{itemize}
    \item Eine andere Bezeichnung für Funktion ist ``Abbildung''.
    \item ''Input'' einer Funktion wird auch Definitionsmenge genannt.
    \item ''Output'' einer Funktion wird auch Wertemenge genannt.
    \item Bei einstelligen Funktionen (also rechtseindeutigen, zweistelligen Relationen)
        ist die Definitionsmenge die Menge aller Elemente,
        die an der ersten Stelle des geordneten Paares stehen.
    \item Bei einstelligen Funktionen (also rechtseindeutigen, zweistelligen Relationen)
        ist die Wertemenge die Menge aller Elemente,
        die an der zweiten Stelle des geordneten Paares stehen.
    \item Bei mehrstelligen Funktionen muss eine Signatur angeben, was zur Definitions-
        bzw. Wertemenge gehört. Z.B. hätte die Funktion $between$,
        die sowohl Vorgänger als auch Nachfolger im Alphabet angibt die folgende Signatur:
        \[ between: 
            \Sigma \backslash \{a,z\}
            \rightarrow
            \Sigma \backslash \{z\} \times \Sigma \backslash \{a\} \]
\end{itemize}

\section{Beweisstrategien \& Beweisskizzen}\label{bew}
