\chapter{Das Hello World der Automatentheorie}
In diesem Kapitel werden wir das \textbf{MATCH}-Problem genauer unter die Lupe nehmen
und dabei unseren ersten Automaten-Typ formal einführen.
Zur Wiederholung die User-Story-Version des \textbf{MATCH}-Problems:
\begin{center}
``Als Einkaufsmanager:in möchte ich, dass nur valide Lieferanten-IDs gespeichert werden
(beginnend mit einem ``L'', gefolgt von einer Zahl),
um Fehler in der Datenbank zu vermeiden.''
\end{center}
Abstrahieren und Formalisieren wir das Problem erscheint folgende informelle Formulierung
als mögliches Ergebnis:
\begin{itemize}
\item \textbf{Gegeben}: Eine Zeichenfolge (String)
\item \textbf{Gesucht}: Entspricht die Zeichenfolge einem bestimmten Muster?
\end{itemize}
Die gesuchte Funktion $f_{MATCH}$ hat als Definitionsmenge also eine Zeichenfolge
über einem Alphabet $\Sigma$ und als Wertemenge die booleschen Werte wahr/falsch,
bzw. 1/0: $f_{MATCH}: \Sigma \rightarrow \{0,1\}$.

Die informelle Version des Problems lässt schon vermuten,
dass seine Berechnung uns nicht vor große theoretische Probleme stellt.
Daher trägt dieses Kapitel auch den Titel ''Das Hello World der Automatentheorie'':
Hello-World-Programme sind einfache Code-Schnipsel,
um ein erstes und einfaches Programm einer Programmiersprache zum Laufen zu bringen
und den Ablauf kennenzulernen (Aufruf des Compilers/Interpreters, grundlegende Syntax, etc).
Analog wollen wir es mit der Automatentheorie halten:
\textbf{MATCH} ist einfach genug, um erste formale Konzepte einzuführen.

\section{Deterministisch Finite Automaten}

In Kapitel \autoref{einleitung} haben wir schon festgehalten:
Probleme lassen sich als als Entscheidungsprobleme von Sprachen formalisieren.
Für Match würde dies bedeuten, dass wir eine Sprache suchen,
die analog zu $f_{MATCH}$ genau die Kombinationen von Zeichenfolgen und booleschen Werten
beinhaltet, die anzeigen, ob die Zeichenfolge nun einem Muster entspricht, oder nicht.
Eine Abkürzung.

Sprach-Generatoren/Sprach-Akzeptoren $\rightarrow$ Einführung DFA.
\section{Reguläre Ausdrücke}
\section{Minimalisierung}
\section{Grenzen endlicher Automaten}
