\chapter{Das Hello World der Automatentheorie}
In diesem Kapitel werden wir das \textbf{MATCH}-Problem genauer unter die Lupe nehmen
und dabei unseren ersten Automaten-Typ formal einführen.
Zur Wiederholung die User-Story-Version des \textbf{MATCH}-Problems:
\begin{center}
``Als Einkaufsmanager:in möchte ich, dass nur valide Lieferanten-IDs gespeichert werden
(beginnend mit einem ``L'', gefolgt von einer Zahl),
um Fehler in der Datenbank zu vermeiden.''
\end{center}
Abstrahieren und Formalisieren wir das Problem erscheint folgende informelle Formulierung
als mögliches Ergebnis:
\begin{itemize}
\item \textbf{Gegeben}: Eine Zeichenfolge (String)
\item \textbf{Gesucht}: Entspricht die Zeichenfolge einem bestimmten Muster?
\end{itemize}
Die gesuchte Funktion $f_{MATCH}$ hat als Definitionsmenge also eine Zeichenfolge
über einem Alphabet $\Sigma$ und als Wertemenge die booleschen Werte wahr/falsch,
bzw. 1/0: $f_{MATCH}: \Sigma \rightarrow \{0,1\}$.

Die informelle Version des Problems lässt schon vermuten,
dass seine Berechnung uns nicht vor große theoretische Probleme stellt.
Daher trägt dieses Kapitel auch den Titel ''Das Hello World der Automatentheorie'':
Hello-World-Programme sind einfache Code-Schnipsel,
um ein erstes und einfaches Programm einer Programmiersprache zum Laufen zu bringen
und den Ablauf kennenzulernen (Aufruf des Compilers/Interpreters, grundlegende Syntax, etc).
Analog wollen wir es mit der Automatentheorie halten:
\textbf{MATCH} ist einfach genug, um erste formale Konzepte einzuführen.



In Kapitel \autoref{einleitung} haben wir schon festgehalten:
Probleme lassen sich als als Entscheidungsprobleme von Sprachen formalisieren.
Für Match würde dies bedeuten, dass wir eine Sprache suchen,
die analog zu $f_{MATCH}$ genau die Kombinationen von Zeichenfolgen und booleschen Werten beinhaltet,
die anzeigen, ob die Zeichenfolge nun einem Muster entspricht oder nicht.

Wenn die Funktion, die wir formalisieren wollen, als Wertemenge die booleschen Werte hat,
können wir allerdings eine kleine Vereinfachung anwenden:
Das Entscheidungsproblem, also die Fragestellung, ob ein Wort zu einer Sprache gehört,
gibt ja schon einen booleschen Wert zurück.
Daher können wir die Wertemenge im Fall von \textbf{MATCH} als redundant ignorieren
und kommen zu folgender Sprache:
Alle Zeichenketten, die einem Muster entsprechen
(z.B. beginnend mit einem ''L'' gefolgt von einer Zahl).
In der Sprache $L_{MATCH}$ wären also z.B. 'L0101' oder 'L111',
nicht aber '1L1' oder 'abc'.

Im folgenden Abschnitt wollen wir uns formale Methoden anschauen,
wie wir eine solche Sprache erkennen können.

\section{Deterministisch Finite Automaten}
Viele formale Sprachen haben unendlich viele Wörter,
so auch $L_{MATCH}$.
Daraus folgt: wir können sie nicht einfach extensiv angeben (''aufzählen''),
sondern müssen Mittel und Wege finden, sie formal zu charakterisieren.
Zwei Methoden bieten sich an:
\begin{enumerate}
    \item Sprach-\emph{Generatoren}, d.h. Formalismen,
        mit denen alle Wörter einer Sprache abgeleitet werden können
        (und nicht mehr).
    \item Sprach-\emph{Akzeptoren}, d.h. Formalismen,
        die für eine gegebenes Wort entscheiden,
        ob es zur Sprache gehört (oder nicht).\footnote{
            Wie wir in \autoref{derBarbierUndDerLuegner} sehen werden,
            ist dieses ''oder nicht'' leider nicht so trivial.}
\end{enumerate}

Sprach-Generatoren sind nicht der Fokus dieses Skriptes, obwohl wir mit den Grammatiken
in \autoref{chmoskyUndGrammatiken} einen solchen Formalismus kennen lernen werden.\footnote{
Einige Lehrbücher der theoretischen Informatik bauen Ihren Stoff didaktisch anhand der Grammatiken
auf (sie erlauben die Sprachen hierarchisch zu klassifizieren).
Wer dieses Skript begleitend z.B. zu \cite{schoening} liest,
sollte diesen Unterschied präsent haben.}
Unser Fokus wird auf den Sprach-Akzeptoren liegen,
da sie sich gut für einen Problem-orientierten Ansatz eignen.

Was für einen Sprach-Akzeptor können wir also für $L_{MATCH}$ heranziehen?
Bühne frei für den Deterministisch Finiten Automaten (kurz: DFA).
Bevor wir DFAs formal einführen, wollen wir ihn informell motivieren: 
Ein Algorithmus, der das \textbf{MATCH}-Problem löst, könnte so aussehen:
\begin{enumerate}
    \item Lese das erste Zeichen $c_1$ des Inputs.
    \item wenn $z_1 \neq $ `$L$', gib 0 aus.
    \item Lese das nächste Zeichen $z_i$ des Inputs.
    \item Wenn $z_i \neq $ `$1$' oder $z_i \neq $ `$0$', gib 0 aus.
    \item Wenn der Input noch nicht zu Ende ist: Gehe zu Schritt 3.
    \item Gib 1 aus.
\end{enumerate}

\section{Reguläre Ausdrücke und Nicht-Determinismus}
\section{Minimalisierung}
\section{Grenzen endlicher Automaten}
