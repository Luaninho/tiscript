\section{Lösungen}

\subsection*{Aufgaben aus Kapitel 2}
\subsubsection*{2.1}

$|\delta| = |\Sigma| * |Z|$\\


\subsubsection*{2.2}

\begin{enumerate}[label=(\alph*)]
    \item $A = [\Sigma, Z, \delta, z_i, E]$ mit:
        \begin{itemize}
            \item $\Sigma = \{a,b\}$
            \item $Z = \{z_1, z_2, z_3, z_4\}$
            \item $\delta = \{
                    [a, z_1, z_2], [b, z_1, z_3],
                    [a, z_2, z_2], [b, z_2, z_2],
                    [a, z_3, z_4], [b, z_3, z_3],
                    [a, z_4, z_4], [b, z_4, z_3]
                \}$
            \item $z_i = z_1$
            \item $E = \{z_1, z_2, z_4\}$
        \end{itemize}

    \item $\epsilon|a(a|b)^*|b(a|b)^*a$

    \item Zum Beispiel: b, bb, bab

    \item A akzeptiert alle Worte, die entweder mit b beginnen oder nicht mit b enden.
\end{enumerate}

\subsubsection*{2.3}
Die Hilfstabelle nach dem letzten Schritt:
\begin{table}[H]
    \centering
    \begin{tabular}{|c|c|c|c|c|c|}
        \hline % Beginn Zeile 1
            \textbackslash
            & $z_0$
            & $z_1$
            & $z_2$
            & $z_3$
            & $z_4$
            \\
        \hline % Beginn Zeile 2
            $z_0$
            & \cellcolor{gray}
            & \cellcolor{gray}
            & \cellcolor{gray}
            & \cellcolor{gray}
            & \cellcolor{gray}
            \\
        \hline % Beginn Zeile 3
            $z_1$
            & x
            & \cellcolor{gray}
            & \cellcolor{gray}
            & \cellcolor{gray}
            & \cellcolor{gray}
            \\
        \hline % Beginn Zeile 4
            $z_2$
            & x
            &
            & \cellcolor{gray}
            & \cellcolor{gray}
            & \cellcolor{gray}
            \\
        \hline % Beginn Zeile 5
            $z_3$
            & x
            & x
            & x
            & \cellcolor{gray}
            & \cellcolor{gray}
            \\
        \hline % Beginn Zeile 6
            $z_4$
            & x
            &
            &
            & x
            & \cellcolor{gray}
            \\
        \hline % Endelinie
    \end{tabular}
    \label{tab:sol23}
\end{table}

Grafik \autoref{fig:sol23} zeigt den resultierenden Automaten.

\begin{figure}[ht] % ’ht’ tells LaTeX to place the figure ’here’ or at the top of the page
\centering % centers the figure
\begin{tikzpicture}
	\node[state, accepting, initial] (z0) {$z_0$};
	\node[state, accepting, right of=z0] (z3) {$z_3$};
    \node[state, below right of=z0] (z124) {$z_{124}$};
	\draw 
        (z0) edge[above] node{0} (z3)
        (z0) edge[bend left, above] node{1} (z124)
        (z3) edge[bend left, right] node{0,1} (z124)
        (z124) edge[bend left, below] node{0,1} (z0)
    ;
\end{tikzpicture}
\caption{Minimierter DFA B}
\label{fig:sol23}
\end{figure}

\subsection*{Aufgaben aus Kapitel 3}

\subsubsection*{3.1}

\begin{itemize}
    \item Lauf auf $01$:

$[
    [z_i, \square, 0, 1],
    [z_1, \square\square, 1, \square],
    [z_2, \square\square 1, \square,\square],$

$
    [z_3, \square\square , 1,\square\square],
    [z_4, \square\square\square , \square,\square],
    [z_a, \square\square\square , \square,\square]
]$
    \item Lauf auf $1$:

$[
    [z_i, \square, 1, \square]
]$
\end{itemize}

\subsubsection*{3.2}
\begin{figure}[ht] % ’ht’ tells LaTeX to place the figure ’here’ or at the top of the page
\centering % centers the figure
\begin{tikzpicture}
	\node[state, initial] (z0) {$z_0$};
	\node[state, right of=z0] (z1) {$z_1$};
	\node[state, right of=z1, accepting] (z2) {$z_2$};
	\draw
    (z0) edge[loop above] node[text width=1cm, align=center]{$*,*,R$} (z0)
	(z0) edge[above] node{$\square$, *, L} (z1)
    (z1) edge[loop above] node[text width=1cm, align=center]{$*,*,L$} (z1)
	(z1) edge[above] node{$\square$, $\square$, R} (z2)
    ;
\end{tikzpicture}
\caption{Turing Maschine M für Übung 3.2}
\label{fig:tm32}
\end{figure}

Die TM aus \autoref{fig:tm32} berechnet das unäre Inkrement,
d.h. gegeben eine unäre Zahl erhöht sie deren Wert um eins.
Die Turingmaschine aus \autoref{fig:tm32alt} berechnet die gleiche Funktion:
\begin{figure}[ht] % ’ht’ tells LaTeX to place the figure ’here’ or at the top of the page
\centering % centers the figure
\begin{tikzpicture}
	\node[state, initial] (z0) {$z_0$};
	\node[state, right of=z0, accepting] (z1) {$z_1$};
	\draw
    (z0) edge[above] node[text width=1cm, align=center]{$*,*,L$} (z1)
	(z1) edge[loop above, above] node{$\square$, *, -} (z1)
    ;
\end{tikzpicture}
\caption{Kleinere Turing Maschine M für Übung 3.2}
\label{fig:tm32alt}
\end{figure}




\subsubsection*{3.3}
Siehe \autoref{fig:tm33} oder \url{http://turingmachinesimulator.com/shared/idlvtuqvpo}.
\begin{figure}[ht] % ’ht’ tells LaTeX to place the figure ’here’ or at the top of the page
\centering % centers the figure
\begin{tikzpicture}
	\node[state, initial] (z0) {$z_0$};
	\node[state, right of=z0] (z1) {$z_1$};
	\node[state, right of=z1] (z2) {$z_2$};
	\node[state, right of=z2] (z3) {$z_3$};
	\node[state, right of=z3] (z4) {$z_4$};
	\node[state, right of=z4, accepting] (z5) {$z_5$};
	\draw
    (z0) edge[above] node{$+,\square,R$} (z1)
	(z1) edge[loop, above] node{$+,+,R$} (z1)
	(z1) edge[above] node{$\square,+,R$} (z2)
	(z2) edge[above] node{$+,+,L$} (z3)
	(z3) edge[loop, above] node{$+,+,L$} (z3)
	(z3) edge[above] node{$\square,\square,R$} (z4)
	(z4) edge[above] node{$+,\square,R$} (z5)
    ;
\end{tikzpicture}
\caption{Turing Maschine für Übung 3.3}
\label{fig:tm33}
\end{figure}

\subsubsection*{3.4}
(a) $M = [\Sigma, \Sigma \cup \square, Z, \delta, z_1, \square, E]$ mit
\begin{itemize}
    \item $\Sigma = \{|,-\}$
    \item $Z = \{z_i, z_l, z_g, z_p, z_m, z_d, z_c, z_n, z_r\}$
    \item $
        \begin{array}{llll}
            \delta = \{&
                [|, z_i, z_i, |, R], &
                [-, z_i, z_i, -, R], &
                [\square, z_i, z_l, \square, L], \\
            &
                [|, z_l, z_m, \square, L], &
                [-, z_l, z_g, |, L], \\
            &
                [|, z_g, z_g, |, L], &
                [\square, z_g, z_p, \square, R], \\
            &
                [|, z_m, z_m, |, L], &
                [-, z_m, z_m, -, L], &
                [\square, z_m, z_d, \square, R], \\
            &
                [|, z_d, z_c, \square, R], \\
            &
                [|, z_c, z_r, |, R], &
                [-, z_c, z_n, |, R], \\
            &
                [\square, z_n, z_n, -, N], \\
            &
                [|, z_r, z_r, |, R], &
                [-, z_r, z_r, -, R], &
                [\square, z_r, z_l, \square, L] \\
            \}
        \end{array}
    $
    \item $E = \{z_n, z_p\}$
\end{itemize}
(b)
$
\begin{array}{llrlrl}
    [ & [ z_i, &    \square, &                  |, &        |-|] & , \\
      & [ z_i, &    \square|, &                 |, &        -|] & , \\
      & [ z_i, &    \square||, &                -, &        |] & , \\
      & [ z_i, &    \square||-, &               |, &        \square] & , \\
      & [ z_i, &    \square||-|, &              \square, &  \square] & , \\
      & [ z_l, &    \square||-, &               |, &        \square\square] & , \\
      & [ z_m, &    \square||, &                -, &        \square\square\square] & , \\
      & [ z_m, &    \square|, &                 |, &        -\square\square\square] & , \\
      & [ z_m, &    \square, &                  |, &        |-\square\square\square] & , \\
      & [ z_m, &    \square, &                  \square, &  ||-\square\square\square] & , \\
      & [ z_d, &    \square\square, &           |, &        |-\square\square\square] & , \\
      & [ z_c, &    \square\square\square, &    |, &        -\square\square\square] & , \\
      & [ z_r, &    \square\square\square|, &   -, &        \square\square\square] & , \\
      & [ z_r, &    \square\square\square|-, &  \square, &  \square\square] & , \\
      & [ z_l, &    \square\square\square|, &   - &         \square\square\square] & , \\
      & [ z_g, &    \square\square\square, &    | &         |\square\square\square] & , \\
      & [ z_g, &    \square\square, &           \square &   ||\square\square\square] & , \\
      & [ z_p, &    \square\square\square, &    | &         |\square\square\square] & ]\\
\end{array}
$

(c) M berechnet die Subtraktion auf unär kodierten ganzen Zahlen.

\subsection*{Aufgaben aus Kapitel 5}
\subsubsection*{5.1}
\begin{figure}[ht] % ’ht’ tells LaTeX to place the figure ’here’ or at the top of the page
\centering % centers the figure
\begin{tikzpicture}
    \node[state, initial] (zi) {$\{z_i\}$};
    \node[state, right of=zi] (z12) {$\{z_1,z_2\}$};
    \node[state, right of=z12, accepting] (z23) {$\{z_2,z_3\}$};
    \node[state, below of=zi,  accepting] (z3) {$\{z_3\}$};
    \node[state, right of=z3, accepting] (z13) {$\{z_1,z_3\}$};
    \node[state, right of=z13] (z1) {$\{z_1\}$};
    \node[state, below of=z13] (ze) {$\emptyset$};
	\draw
    (zi) edge[above] node{0} (z12)
    (zi) edge[left] node{1} (z3)
    (z12) edge[above] node{0} (z23)
    (z12) edge[left] node{1} (z13)
    (z23) edge[loop right] node{0} (z23)
    (z23) edge[left] node{1} (z1)
    (z3) edge[left] node{0,1} (ze)
    (z1) edge[left] node{0} (ze)
    (z1) edge[above] node{1} (z13)
    (z13) edge[left] node{0} (ze)
    (z13) edge[loop left] node{1} (z13)
    ;


\end{tikzpicture}
\caption{Lösungsautomat zu Aufgabe 5.1}
\label{fig:sol51}
\end{figure}
