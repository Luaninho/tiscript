\chapter{Aufwand}

 Was müssen alle Maschinen gemein haben, die $f_{EVEN}$ berechnen?
 Wir müssen verlangen,
 dass jede Maschine eine Eingabe und eine Ausgabe hat,
 mit denen wir $n$ übergeben und den Wahrheitswert ablesen.
 Wie müssten wir den Input und Output der Maschine gestalten,
 d.h. in welche Form müssten wir das \textbf{EVEN}-Problem bringen,
 dass die Maschine es lesen kann und wir den Output wieder verstehen können? 

 Ein zugängliches Modell für den Input ist die Tastatur und der Bildschirm:
 Eine endliche Menge von Zeichen (angezeigt durch Drücken der Knöpfe einer Tastatur)
 werden von der Maschine übersetzt in eine Reaktion der Maschine 
 (angezeigt durch die Pixel einer digitalen Ausgabe).
 Tatsächlich werden die meisten Informationen,
 die eine Maschine oder ein Computer verarbeiten,
 intern in eine sehr kleine Menge an Zeichen übersetzt:
 in Zeichenfolgen bestehend aus Nullen und Einsen.
 Tastaturen und Bildschirme machen uns die Interaktion mit der Maschine einfacher,
 allerdings sind sie für die Maschine selbst nicht nötig.

 Ganz allgemein könnte man also sagen, dass alle Probleme, die ein Computer lösen soll,
sich in diese Form bringen lassen:
 \begin{itemize}
    \item \textbf{Gegeben:} Eine binäre Zeichenfolge.
    \item \textbf{Gesucht:} Eine binäre Zeichenfolge.
 \end{itemize}

 Die Sprache ist also der großen Tabelle nicht unähnlich,
 in der man die ``richtigen'' $n$-Werte ablesen kann.
 Wir wissen, dass diese Sprache unendlich viele Wörter hat und dass es unwirtschaftlich wäre
 die Sprache wirklich aufzuschreiben (wir würden nie fertig werden).
 Perspektivisch müssen wir also nach endlichen Methoden suchen,
 (potentiell) unendliche Sprachen zu verarbeiten.

\section{Algorithmus und Implementierung}
\section{Zeit und Raum}
\section{Big-O: Wachstum}
\section{Messen vs. Beweisen}\label{messenVsBeweisen}
\section{Algorithmen und Datenstrukturen}
