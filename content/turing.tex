\chapter{Alan Turing}\label{turing}

In diesem Kapitel lernen wir ein Problem kennen,
dass nicht DFA-berechenbar ist,
d.h. für dessen formale Sprache kein akzeptierender DFA existiert.
Dies beweisen wir mit Hilfe des Pumping-Lemmas,
das wir informell herleiten.
Mit den Turingmaschinen lernen wir einen Formalismus kennen,
deren Berechenbarkeitsumfang größer ist.
Die Church-Turing-These besagt,
dass es keinen bekannten Berechenbarkeitsformalismus gibt,
der mächtiger ist als der, der Turing-Berechenbarkeit.
Wir schließen das Kapitel ab mit einer formalen Fingerübung:
Es ist möglich Turingmaschinen selbst als Input für Turingmaschinen zu kodieren.
Wir können also fragen, welche Eigenschaften von Turingmaschinen berechenbar sind.

\section{Grenzen endlicher Automaten}\label{pumping}
Wir haben im vorangegangen Kapitel offen gelassen,
ob sich \emph{alle} Probleme als reguläre Sprache darstellen lassen,
haben aber erhebliche Zweifel angemeldet.
In diesem Abschnitt werden wir ein Problem kennenlernen,
dass sich nicht als reguläre Sprache darstellen lässt.
Es ist abgeleitet von U\textsubscript{SORTIMENT}:
``Als Einkaufsmanager:in möchte ich,
dass das Sortiment in allen Filialien identisch ist,
um alle anderen Prozesse darauf ausrichten zu können.``
Die informelle Form
I\textsubscript{SORTIMENT}
wollen wir so angeben:
\begin{itemize}
    \item Gegeben: Zwei Zahlen
    \item Gesucht: Sind die beiden Zahlen identisch?
\end{itemize}
Die formale Sprache $SORTIMENT$ wollen wir noch ein wenig simpler halten:
$\{0^n1^m|n=m\}$.
Hier verstehen wir die konkatenierten Nullen als unäre Repräsentation der ersten Zahl
und die konkatenierten Einsen als unäre Repräsentation der zweiten Zahl.
Ein DFA müsste also entscheiden können, ob die Anzahl der Nullen und Einsen identisch sind.
Da endliche Automaten keine Form des Speichers haben
(die einzige Information, die sich speichern lässt, ist der aktuelle Zustand)
erscheint es intuitiv, dass dieses Problem von keinem DFA gelöst werden kann.
Diese Intuition wollen wir nun etwas formaler greifen,
bevor wir sie nutzen, um zu zeigen, dass $SORTIMENT$ nicht regulär ist.

\subsection{Das Pumping-Lemma}

Wir wollen uns dem Pumping-Lemma mit einer Reihe von Beobachtungen nähern.

\subsubsection{Notwendige und hinreichende Bedingung für Regularität}

Wie wir in \autoref{regex} festgestellt haben, sind alle endlichen Sprachen regulär.
Daher kann man sagen, dass Endlichkeit \emph{hinreichend} für die Regularität einer Sprache ist.
Oder formal ausgedrückt: $L \text{ endlich} \rightarrow L \text{ regulär}$.
Es kann aber auch unendliche Sprachen geben, die regulär sind,
also ist Endlichkeit keine \emph{notwendige} Bedingung für Regularität.

Das Pumping-Lemma dagegen formuliert eine notwendige Eigenschaft für Regularität,
die nicht zureichend ist:
Informell besagt das Lemma,
dass alle regulären Sprachen entweder endlich sind
oder sich auf eine bestimmte Weise ``aufpumpen`` lassen (daher der Name).
 $L \text{ regulär} \rightarrow L \text{ endlich} \vee L \text{ ist aufpumpbar}$.
Da es auch nicht-reguläre Sprachen gibt, die in unserem Sinne ``aufpumpbar'' sind,\footnote{
    Siehe \cite{schoening} 40 für ein Beispiel.}
kann man das Pumping-Lemma nicht für einen direkten Beweis der Regularität nutzen.
Momentan suchen wir aber nach einer Methode zu zeigen,
dass $SORTIMENT$ \emph{nicht} regulär ist
und dafür kann man das Pumping-Lemma nutzen:
Ist eine Sprache weder endlich, noch in unserem Sinne ``aufpumbar'',
dann kann sie nicht regulär sein.
Wir wollen uns diesem Lemma und seiner Prädikaten-logischen Form anhand eines Beispiels nähern.

\subsubsection{uvw-Zerlegung aufpumbarer Worte}

Als Beispiel dafür, was wir mit ``aufpumpbar'' meinen,
wollen wir die Sprache L betrachten, die durch den DFA in \autoref{fig:dfapump}
oder den regulären Ausdruck $10^*1|0000$ gegeben ist.
Der DFA zu L hat zwei Arten von Läufen zu einem Endzustand:
\begin{enumerate}
    \item Einen ``Einzellauf'' via $z_i, z_3, z_4, z_5$ zu $z_6$.
        Auf diesem Pfad wird genau ein Wort akzeptiert: $0000$.
    \item Einen ``Zirkellauf'' via $z_i, z_1^*, z_2$.
        Auf diesem Pfad werden unendlich viele Wörter akzeptiert ($11, 101, 10001, \ldots$).
        Dieser Zirkellauf enthält, daher der Name, einen Zirkel, bzw. einen Loop.
\end{enumerate}

Unsere \textbf{erste} Beobachtung:
Ab einer gewissen Länge von Wörtern in L kommen wir nur über den Zirkellauf zu einem
Endzustand.
Wir wollen diese Länge $n$ nennen. Im Falle von L ist die Länge 5.

Unsere \textbf{zweite} Beobachtung:
Für ein Wort $x \in L$ mit $|x| \geq n = 5$ gilt: Wir können es in drei Teile zerlegen
$x = uvw$:
\begin{enumerate}
    \item $u$ ist der Teil des Wortes,
        dessen Schnappschussfolge mit $s_u = [u,z_1]$ endet.
        $z_1$ ist der Zustand am \emph{Ende des Loops}.\\
        Ein Merkwort für diesen Teil: ``\textbf{u}nterwegs''.
    \item $v$ ist der Teil des Wortes,
        dessen Schnappschussfolge von \emph{variabler Länge} ist.
        Formal: $|v| \in \mathbb{N}$.\\
        Ein Merkwort für diesen Teil: ``\textbf{v}erweile''.
    \item $w$ ist der Teil des Wortes,
        dessen Schnappschussfolge vom Ende des Loops \emph{zum Endzustand} führt.\\
        Ein Merkwort für diesen Teil: ``\textbf{w}eiter''.
\end{enumerate}
Hier wird klar, was wir mit ``aufpumpen'' meinen:
Wir können über die variable Länge von $v$ L auf unendlich viele Worte aufpumpen
(beziehungsweise da $n = 5$ auch auf $11$, $101$, oder $1001$ ``komprimieren'').

Eine \textbf{dritte} Beobachtung:
$|v| \geq 1$, da $n = 5$ erzwingt,
dass wir dreimal den Loop im Lauf ``passieren''.

Eine \textbf{vierte} Beobachtung:
$|uv| \leq n$, in unserem Falle ist $|uv| = 4$ ($uv = 1000$).
\begin{figure}[ht] % ’ht’ tells LaTeX to place the figure ’here’ or at the top of the page
\centering % centers the figure
\begin{tikzpicture}
	\node[state, initial] (zi) {$z_i$};
        \node[state, above right of=zi] (z1) {$z_1$};
        \node[state, accepting, right of=z1] (z2) {$z_2$};

        \node[state, below right of=zi] (z3) {$z_3$};
        \node[state, right of=z3] (z4) {$z_4$};
        \node[state, right of=z4] (z5) {$z_5$};
        \node[state, accepting, right of=z5] (z6) {$z_6$};

        \node[state, right of=zi, above of=z4] (z7) {$z_7$};
	\draw
        (zi) edge[below] node{0} (z3)
        (zi) edge[above] node{u: 1} (z1)

        (z1) edge[loop above] node{v: 0} (z1)
        (z1) edge[above] node{w: 1} (z2)

        (z2) edge[above] node{0,1} (z7)

        (z3) edge[above] node{0} (z4)
        (z3) edge[above] node{1} (z7)

        (z4) edge[above] node{0} (z5)
        (z4) edge[above] node{1} (z7)

        (z5) edge[above] node{0} (z6)
        (z5) edge[above] node{1} (z7)

        (z6) edge[above] node{0,1} (z7)


        (z7) edge[loop above] node{0,1} (z7)
    ;
\end{tikzpicture}
\caption{DFA für $10^*1|0000$}
\label{fig:dfapump}
\end{figure}

\noindent
Wir können diese Eigenschaft für L etwas formaler fassen:
\begin{multline*}
    \phi(L) =
        \exists n \forall x (
        x \in L
        \wedge  |x| \geq n \rightarrow \\
        \exists u,v,w
        (
            x = uvw
            \wedge  |v| \geq 1
            \wedge  |uv| \leq n
            \wedge  \forall i \in \mathbb{N} (uv^iw \in L)
        )
    )
\end{multline*}

\noindent
Diese Aussage trifft auf jeden Fall auf unser L = $10^*1|0000$ zu.
Wir können tatsächlich ohne Beschränkung der Allgemeinheit diese Eigenschaft $\phi$ auf
\emph{alle} regulären Sprachen anwenden:\footnote{
    Der Beweis in \cite{schoening} 39 argumentiert alternativ,
    dass man $n$ durch die Anzahl der Zustände finden kann,
    ignoriert aber den Fall der endlichen regulären Sprachen.
    Dies lässt sich aber leicht ergänzen.
    Er ist in jedem Fall eine Lektüre wert.}
\begin{enumerate}
    \item Ist $L$ endlich, dann lassen sich die vier Beobachtungen ``nullerweise'' erfüllen,
        das heißt man wähle $n$ so,
        dass es länger ist, als das längste Wort in L.
    \item Ist $L$ so geschaffen, dass der Weg zum Loop ``leer'' ist,
        also $|u| = 0$,
        weil der Loop im Anfangszustand endet,
        gelten die vier Beobachtungen dennoch.
    \item Ist $L$ so geschaffen, dass der Weg vom Loop zum Endzustand ``leer'' ist,
        also $|w| = 0$,
        weil der Loop selbst den Endzustand enthält,
        gelten die vier Beobachtungen dennoch.
    \item Hat $L$ mehr als einen Zirkel auf dem Weg zu einem Endzustand,
        finden wir für jedes Wort eine solche Zerlegung auf einem der Zirkel.
\end{enumerate}
$\phi$ bietet eine formalere Beschreibung der Eigenschaft
``$L \text{ endlich } \vee L \text{ ist aufpumpbar}$''.
\subsection{Beweis der Nicht-Regularität}
Das Pumpinglemma können wir nun wie folgt formulieren:
\[
    L \text{ regulär} \rightarrow \phi(L)
\]
$\phi$ ist wie bereits geschrieben nur eine \emph{notwendige},
keine \emph{zureichende} Bedingung für die Regularität einer Sprache.
Durch die Kontraposition des Pumpinglemmas, können wir es aber nutzen,
um zu beweisen, dass eine Sprache L \emph{nicht} regulär ist:
\[
    \neg \phi(L) \rightarrow L \text{ nicht regulär}
\]
Buchstabieren wir $\neg \phi(L)$ aus:
\begin{multline*}
    \forall n \exists x
    (
        x \in L
        \wedge  |x| \geq n
        \rightarrow \\
        \forall u,v,w (
            x = uvw
            \wedge  |v| \geq 1
            \wedge  |uv| \leq n
            \rightarrow  \exists i \in \mathbb{N} (uv^iw \notin L)
        )
    )
\end{multline*}
Hieraus ergibt sich eine Blaupause,
mit der wir die Nicht-Regularität einer Sprache beweisen können:
\begin{enumerate}
    \item Sei ein beliebiges $n$ gegeben.
    \item Wähle ein Wort $x \in L$ mit $|x| \geq n $.
    \item Nimm an, dass die Bedingungen $x = uvw$, $|v| \geq 1$, $|uv| \leq n$ wahr sind.
    \item Zeige, dass es ein $i \in \mathbb{N}$ gibt, für das gilt: $uv^iw \notin L$
\end{enumerate}

Auf unserem Problem SORTIMENT ($\{0^n1^m|n=m\}$) angewandt:\footnote{Der Beweis ist direkt aus \cite{schoening} 41 übernommen.}
Sei $x = 0^n1^n$, wegen $|v| \geq 1$ ist $v$ nicht leer,
wegen $|uv| \leq n$ besteht $v$ nur aus Nullen.
Mit $i = 0$ gilt: $uw \notin L$, da $uw = 0^{n-|v|}1^n$.
Also ist die Sprache nicht regulär und kann nicht von einem DFA erkannt werden.

\subsection{Grenzen der DFA-Berechenbarkeit}
$SORTIMENT$ ist ein Problem, das uns \emph{intuitiv} berechenbar erscheint.
Aber DFAs als Formalismus, um Algorithmen anzugeben,
sind beweisbar nicht mächtig genug,
um $SORTIMENT$ zu berechnen.

Der wesentliche Grund hierfür: ein DFA hat keinen ``Speicher'',
d.h. Algorithmen, die mehr brauchen, als den aktuellen Zustand,
in dem sie sich gerade befinden,
können auf einem DFA nicht berechnet werden.
Nichtsdestotrotz ist ein DFA mächtig genug,
um für ein gegebenes $n \in \mathbb{N}$ die Sprache
$L = \{0^i1^i| i \leq n\}$ zu erkennen!
Dies ist darin begründet,
dass $L$ endlich ist.\footnote{
    \cite{aul} ist ein lesenswertes Studienbuch, dass diesen Gedanken formal ausbuchstabiert.
}

Im nächsten Abschnitt lernen wir einen Formalismus kennen,
der mächtiger ist, als ein DFA.

\section{Turingmaschinen}
\subsection{Wer war Alan Turing?}
Der Namensgeber des aktuellen Kapitels ist es wert,
dass wir ein paar autobiographische Notizen über ihn verlieren.
Alan Turing (1912-1954) war ein britischer Mathematiker,
dessen Bedeutung für die Informatik schwer zu überschätzen ist.
Hier nur eine Auswahl von Aspekten seines Werkes,
die nicht oder nicht unmittelbar mit dem Thema unseres Skriptes zu tun haben:
\begin{itemize}
    \item Der \emph{Turing-Test} ist ein viel diskutierter Gradmesser,
        inwiefern wir Maschinen ``künstliche'' Intelligenz zusprechen,
        grob gesagt dann, wenn sich die Interaktion der Maschine für Proband:innen
        für diese nicht signifikant von der Interaktion mit anderen Menschen unterscheidet.
    \item Die \emph{Turing-Bombe}, eine elektro-magnetische Maschine,
        die von Alan Turing konzipiert wurde,
        half den zweiten Weltkrieg zu entscheiden,
        da sie es ermöglichte,
        deutsche Funksprüche zu entschlüsseln,
        die mit der Enigma-Verschlüsselungsmaschine verschlüsselt wurden.
    \item Der \emph{Turing-Award} gilt als die höchste Auszeichnung der Informatik
        (``Nobelpreis der Informatik'') und wird jährlich von der
        Association for Computing Machinery (ACM) verliehen.
\end{itemize}
Es gibt noch mannigfaltige Gründe,
sich für das Leben von Alan Turing zu interessieren.
Besonders tragisch ist die Tatsache,
dass Turing früh verstarb:
Er wurde aufgrund seiner Homosexualität (seinerzeit eine Straftat in Großbrittanien)
1952 zur chemischen Kastration verurteilt.
1954 nahm er sich unter ungeklärten Umständen das Leben.

\subsection{Was ist eine Turingmaschine?}

Wir führen den Begriff der Turingmaschine analog zum DFA ein.
Eine (deterministische) Turingmaschine TM ist ein 7-Tupel
$TM = [Z, \Sigma, \Gamma, \delta, z_0, \square, E]$:
\begin{itemize}
    \item $Z$ ist die Menge der Zustände.
    \item $\Sigma$ ist das Eingabealphabet.
    \item $\Gamma$ ist das Arbeitsalphabet. Es gilt: $\Sigma \subsetneq \Gamma$.
    \item $\delta: \Gamma \times Z \rightarrow Z \times \Gamma \times \{R,L,-\}$
        ist die \emph{partielle} Übergangsfunktion.
    \item $z_0$ ist der Startzustand.
    \item $\square$ ist das Blank.
        Es gilt $\square \in \Gamma \wedge \square \notin \Sigma$.
    \item E ist die Menge der Endzustände.
\end{itemize}

\noindent
Eine Turingmaschine funktioniert intuitiv so:
\begin{itemize}
    \item Zusätzlich zum DFA hat sie ein unendliches, beschreibbares Band.
    \item Der Input steht vor dem Lauf auf dem Band.
        Ist ein Input mehrstellig, sind die Werte durch Blanks ($\square$) getrennt.
        Auf allen anderen Position auf dem Band sind Blanks.
    \item Der ``Zeiger'' der Turingmaschine zeigt vor dem Lauf auf das erste Zeichen des Inputs.
    \item Startet die Turingmaschine,
        so liest sie das erste Zeichen
        und führt die folgenden drei Dinge wie von $\delta$ angegeben aus:
        \begin{enumerate}
            \item Das Zeichen wird überschrieben oder bleibt identisch.
            \item Der Zustand wechselt oder bleibt gleich.
            \item Der Zeiger bewegt sich nach
                rechts (R),
                links (L)
                oder bleibt auf der Position stehen (-).
        \end{enumerate}
    \item Die Turingmaschine stoppt,
        wenn $\delta$ undefiniert ist ($\delta$ ist partiell).
    \item Der Output der Turing-Maschine ist alles von der Zeigerposition an rechts.
\end{itemize}

Ebenfalls analog zum DFA lassen sich die folgenden Begriffe definieren:

Ein \emph{Schnappschuss} $s$ einer Turingmaschine TM ist ein 4-Tupel $[z,u,v,w]$:
\begin{itemize}
    \item $z$ ist der aktuelle Zustand.
    \item $u$ ist der Bandinhalt links vom Zeiger\\(ohne die Menge an unendlichen Blanks).
    \item $v$ ist das Zeichen, auf das der Zeiger aktuell zeigt.
    \item $w$ ist der Bandinhalt rechts vom Zeiger\\(ohne die Menge an unendlichen Blanks).
\end{itemize}
Ein Tupel $S$ von Schnappschüssen heißt \emph{Schnappschussfolge},
wenn für je zwei aufeinanderfolgende Schnappschüsse $s_n$ und $s_{n+1}$ gilt:
\begin{itemize}
    \item $z_{n+1} = \delta(v, z_n)_0$
    \item $u_{n+1} = \begin{cases}
            tw'(u_n, 0, |u_n|-1) & \text{wenn } \delta(v_n, z_n)_2 = L\\
            u_n                  & \text{wenn } \delta(v_n, z_n)_2 = -\\
            u_n\delta(v_n,z_n)_1 & \text{wenn } \delta(v_n, z_n)_2 = R
        \end{cases}$
    \item $v_{n+1} = \begin{cases}
            tw'(u_n, |u_n|-1, 1) & \text{wenn } \delta(v_n, z_n)_2 = L\\
            \delta(v_n, z_n)_1   & \text{wenn } \delta(v_n, z_n)_2 = -\\
            tw'(w_n, 0, 1)       & \text{wenn } \delta(v_n, z_n)_2 = R
        \end{cases}$
    \item $w_{n+1} = \begin{cases}
            \delta(v_n,z_n)_1w_n & \text{wenn } \delta(v_n, z_n)_2 = L\\
            w_n                  & \text{wenn } \delta(v_n, z_n)_2 = -\\
            tw'(w_n, 1, |w_n|-1) & \text{wenn } \delta(v_n, z_n)_2 = R
        \end{cases}$
\end{itemize}
\noindent
$tw'$ ist hierbei eine angepasste Version der Teilwortfunktion $tw$ aus \autoref{words}:
\begin{itemize}
    \item $tw'(x, 0, -1) = \square$
    \item $tw'(x, -1, 1) = \square$
    \item $tw'(x, 1, -1) = \square$
    \item $tw'(x, 0, 1) = \square$, wenn $|x| = 0$ (nicht mehr $\epsilon$)
    \item Ansonsten ist $tw'(x, n, m) = tw(x, n, m)$ für $n, m \in \mathbb{N}$.
\end{itemize}
Diese Anpassung ist nötig, damit wir u und w bei Bedarf mit Blanks ``auffüllen'' können.
\\

\noindent
Die Länge $len(S)$ einer Schnappschussfolge S entspricht der Länge des Tupels.
\\

\noindent
Eine Schnappschussfolge S heißt
\emph{Lauf} für $x$ auf einer Turingmaschine $tm$ $l_{x, tm}$
wenn gilt:
\begin{itemize}
    \item $s_0 = [z_i, \square, tw'(x,0,1), tw'(x,1,|w|-1)]$
    \item $s_{len(S)} = [z \in Z, u \in \Gamma^*, v \in \Gamma, w \in \Gamma^*]$
        mit $\delta(v, z)$ undefiniert.
\end{itemize}

\noindent
Ein Lauf $l_{x, tm}$ heißt \emph{akzeptierend},
wenn $(s_{len(l_{x, tm})})_0 \in E$.
\\

\noindent
Die Sprache $TM(L)$, die von einer Turingmaschine akzeptiert wird, ist definiert als
$TM(L) =
\{x|
    x = yz
    \wedge
        \exists s(
            s = (l_{y,tm})_{len(l_{y,tm})}
            \wedge z = s_2s_3
            \wedge s_0 \in E
        )
\}$.
Also die Worte,
die aus der Konkatenation des Inputs der Turingmaschine
mit dem Bandinhalt rechts vom Zeiger nach einem akzeptierenden Lauf entstehen.

\section{Turing-Berechenbarkeit}

Turingmaschinen führen genauso wie DFAs zu einem Berechenbarkeitsbegriff:
Ein Problem ist \emph{Turing-berechenbar},
wenn es eine Turingmaschine gibt,
die die zugehörige formale Sprache erkennt.
In anderen Worten:
Das Problem, das von der formalen Sprache repräsentiert wird,
wird durch den Algorithmus, der durch die Turingmaschine implementiert wird,
gelöst.

Wir können also sagen, dass sowohl DFAs, wie auch Turingmaschinen
Berechenbarkeits\emph{modelle} sind.
Die Güte dieser Modelle können wir beurteilen, 
indem wir uns ansehen, wie sie unseren intuitiven Berechenbarkeitsbegriff abbilden:
Wir wissen bereits, dass es intuitive berechenbare Probleme gibt,
die nicht DFA-berechenbar sind.
Es bleiben allerdings folgende offenen Fragen:

\begin{itemize}
    \item Sind alle DFA-berechenbaren Probleme auch Turing-berechenbar?
        Also, werden alle regulären Sprachen auch von Turingmaschinen erkannt?
        Wenn dies der Fall ist,
        dann ist die Menge der Turing-berechenbaren Probleme
        eine Obermenge der DFA-Berechenbarkeit;
        wir nennen diese Beziehung zwischen
        der Turing-Berechenbarkeit und der DFA-Berechenbarkeit
        dann \emph{mindestens genauso mächtig}.
    \item Gibt es Probleme, die nicht DFA-berechenbar, aber Turing-Berechenbar sind?
        Deren formale Sprache also \emph{nicht} von einem DFA,
        aber von einer Turing-Maschine erkannt werden?
        Wenn dies der Fall ist,
        dann ist die Menge der Turing-berechenbaren Probleme
        eine \emph{echte} Obermenge der DFA-Berechenbarkeit,
        also ist die Turing-Berechenbarkeit \emph{mächtiger} als die DFA-Berechenbarkeit.
    \item Gibt es ein Berechenbarkeitsmodell, das mächtiger ist, als die Turingmaschinen?
        Die Antwort auf diese Frage führt uns zur Church-Turing-These.
\end{itemize}

\subsection{Simulation eines DFAs mit einer Turingmaschine}

Das wesentliche formale Werkzeug,
um Berechenbarkeitsmodelle miteinander zu vergleichen,
ist die \emph{Simulation}.
Wir simulieren ein Berechenbarkeitsmodell A in einem anderen B,
wenn wir zeigen können, dass jeder Schritt in A in B simuliert werden kann,
ohne dass sich das Ergebnis ändert.

Zum Beispiel können wir zeigen,
dass ein DFA in einer Turingmaschine simuliert werden kann:

Sei ein DFA $A = [Z_A, \Sigma, \delta_A, z_i, E_A]$ gegeben.
Dann können wir A mit der folgenden Turingmaschine TM simulieren:
$TM = [Z_A, \Sigma, \Sigma \cup \{\square\}, \delta_{TM}, z_i, \square, E_A]$,
wobei gilt:
$\delta_{TM}(a, z) = [\delta_A(a, z), a, R]$,
das heißt, wir lassen den Zeiger einfach das Wort entlang laufen,
bis wir auf ein Blank stoßen.
Also verhält sich $TM$ exakt wie $A$.

Da die TM in jedem Fall das Wort bis zum letzten Zeichen liest
($\delta_A(\square, z)$ ist nicht definiert sein,
also ist $\delta_{TM}(a,z)$ ebenfalls nicht definiert).
und am Ende entweder in einem Endzustand oder einem Nicht-Endzustand von A ist,
akzeptiert sie ein Wort genau dann, wenn auch A das Wort akzeptiert. 
Es gilt also: $L(TM) = L(A)$ (da der Inhalt des Bandes rechts vom Zeiger immer leer ist).

\subsection{Turing-Berechenbarkeit ist echt mächtiger als DFA-Berechenbarkeit}
Das beste Beispiel um zu beweisen,
dass es Probleme gibt, die von einer Turingmaschine erkannt werden,
aber nicht von einem DFA ist $SORTIMENT = \{0^n1^n| n \in \mathbb{N}\}$,
da wir hier in \autoref{pumping} schon bewiesen haben,
dass $SORTIMENT$ nicht DFA-berechenbar ist.

Die Turingmaschine aus \autoref{fig:tmsortiment} akzeptiert SORTIMENT.

\begin{figure}[ht] % ’ht’ tells LaTeX to place the figure ’here’ or at the top of the page
\centering % centers the figure
\begin{tikzpicture}
	\node[state, initial] (zi) {$z_i$};
	\node[state, right of=zi] (z1) {$z_1$};
	\node[state, right of=z1] (z2) {$z_2$};
	\node[state, right of=z2] (z3) {$z_3$};
	\node[state, right of=z3] (z4) {$z_4$};
	\node[state, accepting, below of=z2] (za) {$z_a$};
	\node[state, above of=z3] (z5) {$z_5$};
	\node[state, above of=z1] (z6) {$z_6$};
	\draw
        (zi) edge[above] node{0,$\square$,R} (z1)
        (zi) edge[below left] node{$\square$,$\square$,-} (za)
        (z1) edge[loop above] node{0,0,R} (z1)
        (z1) edge[above] node{1,1,R} (z2)
        (z2) edge[loop above] node{1,1,R} (z2)
        (z2) edge[above] node{$\square$,$\square$,L} (z3)
        (z3) edge[above] node{1,$\square$,L} (z4)
        (z4) edge[below right] node{$\square$,$\square$,-} (za)
        (z4) edge[right] node{1,1,L} (z5)
        (z5) edge[loop above] node{1,1,L} (z5)
        (z5) edge[above] node{0,0,L} (z6)
        (z6) edge[loop above] node{0,0,L} (z6)
        (z6) edge[left] node{$\square$,$\square$,R} (zi)
    ;
\end{tikzpicture}
\caption{Turingmaschine, die $SORTIMENT$ akzeptiert}
\label{fig:tmsortiment}
\end{figure}

Die folgende Schnappschussfolge zeigt, wie diese Turingmaschine operiert:\footnote{
    Für Turingmaschinen gibt des diverse online Tools mit denen man Testläufe durchführen kann:
    z.B. \url{https://turingmachinesimulator.com/shared/wwydbamznw}
}
\[
    01: [[z_i, \epsilon, 0, 1],
         [z_1, \epsilon, 1, \square],
         [z_2, 1, \square, \square],
         [z_3, \square, 1, \square],
         [z_4, \square, \square, \square],
         [z_a, \square, \square, \square]]
\]

%TODO: Schnappschussfolge für $0011$ und $011$

\subsection{Turing-Vollständigkeit}\label{turingVollstaendigkeit}

Wir haben nun gezeigt,
dass alle Algorithmen,
die auf einem DFA realisiert werden können,
auch auf einer Turingmaschine realisiert werden können.
Wir haben auch gezeigt,
dass wir mit Turingmaschinen Algorithmen implementieren können,
die auf einem DFA aus beweisbaren Gründen nicht möglich sind.
In diesem Abschnitt gehen wir noch einen Schritt weiter,
indem wir die \emph{Church-Turing-These} einführen:
\textbf{Die Klasse der \emph{intuitiv} berechenbaren Probleme
ist identisch mit der Klasse der Turing-berechenbaren Probleme.}
Die Church-Turing-These besagt,
dass es keinen Formalismus gibt,
der Berechnungs-mächtiger ist als Turingmaschinen.
Ein wichtiges Merkmal dieser These ist ihre \emph{empirische} Natur,
das bedeutet: sie lässt sich nicht beweisen, nur widerlegen.

Wie sähe eine Widerlegung aus?
Man müsste einen Formalismus finden,
der ein Problem löst,
das beweisbar auf einer Turingmaschine nicht gelöst werden kann,
der aber eine Turingmaschine simulieren kann (so wie die Turingmaschine den DFA simuliert).
Ein solcher Formalismus wäre echt mächtiger als die Turingmaschine.
Bisher ist aber nicht gelungen einen solchen Formalismus zu finden.
Für die folgenden Formalismen wurde gezeigt,
dass sie \emph{Turing-vollständig} sind,
das bedeutet:
Man kann auf ihnen eine Turingmaschine simulieren.
Es gilt aber auch,
dass eine Turingmaschine den Formalismus simulieren kann,
sie sind also von gleicher Berechnungsstärke:
\begin{itemize}
    \item While-Progamme (siehe \cite{schoening}, 100-108)
    \item Goto-Programme (siehe \cite{schoening}, 100-108)
    \item $\mu$-rekursive Funktionen (siehe \cite{schoening}, 109-116)
    \item Abacus-Maschinen (siehe \cite{cal}, 45-62)
    \item Jede bekannte Allzweck-Programmiersprache
\end{itemize}


Die folgenden Formalismen sind nicht Turing-vollständig,
können aber von einer Turingmaschine simuliert werden:
\begin{itemize}
    \item Primitiv-rekursive Funktionen (siehe \cite{schoening} 109-115)
    \item Loop-Programme (siehe \cite{schoening} 100-102)
    \item Keller-Automaten (siehe \autoref{keller})
\end{itemize}
Ein Beispiel für eine Funktion,
die turing-berechenbar, aber nicht primitiv-rekursiv ist bzw. Loop-berechenbar
ist die \emph{Ackermann-Funktion} (siehe \cite{schoening}, 116-122).

Wir schlussfolgern aus der Annahme,
dass die Church-Turing-These wahr ist:
Alle Aussagen bzgl. der Komplexität einer Berechnung und der Berechenbarkeit eines Problems
lassen sich an Turing-Maschinen zeigen.
Daher kann man sagen,
dass die Turingmaschinen die \emph{lingua franca} der theoretischen Informatik sind.

\section{Turingmaschinen: Algorithmus und formale Sprache}
Wir haben Turingmaschinen als primäres Werkzeug kennengelernt,
um Algorithmen für die Fragestellungen der theoretischen Informatik zu implementieren.
Aus der Definition des Algorithmus in \autoref{algoundimpl}
können wir vier der fünf Kriterien als erfüllt ansehen:
\begin{itemize}
    \item Turingmaschinen sind wie DFAs formal spezifiziert und damit unmissverständlich.
    \item Turingmaschinen haben einen Input (Bandinhalt vor dem Start)
    \item Turingmaschinen haben einen vom Input abhängigen Output
        (Bandinhalt vom Zeiger ab rechts nach dem Lauf)
    \item Turingmaschinen sind wie DFAs realisierbar,
        damit sind auch die implementierten Algorithmen durchführbar.
\end{itemize}
Der einzig unerfüllte Punkt:
Wir können von Turingmaschinen nicht mit Sicherheit sagen,
ob sie nach endlichen Schritten terminieren.
Diese Diskussion schieben wir aber bis \autoref{derBarbierUndDerLuegner}.
Hier halten wir fest: Turingmaschinen repräsentieren in jedem Fall \emph{Berechnungsmethoden}.

Ein letzter Twist bleibt aber für dieses Kapitel:
Turingmaschinen lassen sich auch als formale Sprache repräsentieren,
d.h. wir können den 7-Tupel zum Beispiel in eine Binärrepräsentation bringen.
Diese Tatsache öffnet das Tor zu einer Reihe von möglichen Fragestellungen:
\begin{itemize}
    \item Welche Eigenschaften einer Turingmaschine kann man ``errechnen'',
        indem man einer Turingmaschine diese Turingmaschine als Input gibt?
    \item Kann man Turingmaschinen auch ``virtualisieren'' also eine Universale
        Turingmaschine angeben, die eine Turingmaschine als Input bekommt,
        diese ``lädt'' und entsprechend ausführt?
\end{itemize}
Die erste Frage wird wiederum in \autoref{derBarbierUndDerLuegner} aufgegriffen.
Für die zweite Frage können wir hier schon ein Aussage treffen:
Eine solche Universal-Turingmaschine gibt es (siehe \cite{CC} 20/21).
Damit verschwimmen die Grenzen zwischen Algorithmus und formaler Sprache etwas,
ungefähr in dem Maße,
wie Virtualisierung und Infrastructure-as-Code in modernen IT-Systemen
den Unterschied zwischen Hard- und Software verschwimmen lassen.

\subsection*{Aufgaben}

\begin{enumerate}
    \item Gib zwei verschiedene Läufe für Wörter auf der Turingmaschine aus
        \autoref{fig:tmsortiment} an.
    \item Gegeben sei die folgende Turingmaschine
        $M = [Z, \{*\}, \{*, \square\}, \delta, z_0, \square, \{z_2\}]$,
        mit $Z = \{z_0, z_1, z_2\}$ und \\
        $\delta = \{
            [*,z_0,z_0,*,R],
            [\square,z_0,z_1,*,L],
            [*,z_1,z_1,*,L],
            [\square,z_1,z_2,\square,R]
        \}$.
        Gib M in grafischer Darstellung an und beschreibe, welche Funktion M berechnet.
        Findest Du eine ``kleinere'' Turingmaschine (weniger Zustände, kleineres $\delta$),
        welche die gleiche Funktion berechnet?
    \item Gib eine Turingmaschine an, die auf dem Alphabet $\Sigma = \{+\}$,
        die unäre Addition realisiert.
        Man kann davon ausgehen, dass der Input (zwei Zahlen in unärer Darstellung)
        mit einem $\square$ getrennt beim Start auf dem Band der Turingmaschine steht.
        In unärer Kodierung gilt: $0 = +, 1 = ++, 2 = +++, \ldots$
    \item Gegeben sei die Turingmaschine M wie in \autoref{fig:tm34}.
        \begin{enumerate}
            \item Geben Sie M in der Sprache der Mengenlehre an.
            \item Geben Sie den Lauf von M auf dem Input $||-|$ an.
            \item Nehmen Sie an, M startet auf einem Band mit einem Input,
                der dem regulären Ausdruck $||*-||*$ entspricht.
                Beschreiben Sie, welche Funktion M berechnet, wenn Sie die Anzahl Striche als
                unäre Kodierung für die natürlichen Zahlen interpretieren.
        \end{enumerate}
    \item Geben sie eine schematische Zeichnung für eine Turingmaschine $M_{bininc}$ an,
        die als Input eine valide Binärzahl bekommt (ohne führende Nullen) und als
        Output die um eins erhöhte Binärzahl auf dem Band stehen hat.
\end{enumerate}

\begin{figure}[ht] % ’ht’ tells LaTeX to place the figure ’here’ or at the top of the page
\centering % centers the figure
\begin{tikzpicture}
	\node[state, initial] (zi) {$z_i$};
	\node[state, right of=zi] (zl) {$z_l$};
	\node[state, above of=zl] (zg) {$z_g$};
	\node[state, accepting, right of=zg] (zp) {$z_p$};
	\node[state, right of=zl] (zm) {$z_m$};
	\node[state, right of=zm] (zd) {$z_d$};
	\node[state, below of=zd] (zc) {$z_c$};
	\node[state, accepting, right of=zc] (zn) {$z_n$};
	\node[state, left of=zc] (zr) {$z_r$};
	\draw (zi) edge[loop above] node[text width=1cm, align=center]{$|,|,R$\\-,-,R} (zi)
	(zi) edge[above] node{$\square$, $\square$, L} (zl)
	(zl) edge[left] node{$-,|,L$} (zg)
	(zg) edge[loop left] node{$|,|,L$} (zg)
	(zg) edge[above] node{$\square$,$\square$,R} (zp)
	(zl) edge[above] node{$|,\square$,L} (zm)
	(zm) edge[loop above] node[text width=1cm, align=center]{$|,|,L$\\-,-,L} (zm)
	(zm) edge[above] node{$\square$,$\square$,R} (zd)
	(zd) edge[right] node{$|,\square$,R} (zc)
	(zc) edge[above] node{$-,|,L$} (zn)
	(zn) edge[loop right] node{$\square$,-,N} (zn)
	(zc) edge[above] node{$|,|,R$} (zr)
	(zr) edge[loop left]  node[text width=1cm, align=center]{$|,|,R$\\-,-,R} (zr)
	(zr) edge[left] node{$\square$,$\square$,L} (zl)
    ;
\end{tikzpicture}
\caption{Turing Maschine M für Übung 3.3}
\label{fig:tm34}
\end{figure}

