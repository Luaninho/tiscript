\chapter{Alan Turing}\label{turing}
\section{Grenzen endlicher Automaten}\label{pumping}
Wir haben im vorangegangen Kapitel offen gelassen,
ob sich \emph{alle} Probleme als reguläre Sprache darstellen lassen,
haben aber erhebliche Zweifel angemeldet.
In diesem Abschnitt werden wir ein Problem kennenlernen,
dass sich nicht als reguläre Sprache darstellen lässt.
Es ist abgeleitet von U\textsubscript{SORTIMENT}:
Als Einkaufsmanager:in möchte ich,
dass das Sortiment in allen Filialien identisch ist,
um alle anderen Prozesse darauf ausrichten zu können.


\section{Turingmaschinen}
\subsection{Quantifikation von Aufwand}
\section{Turing-Berechenbarkeit}
\subsection{Turing-Vollständigkeit}\label{turingVollstaendigkeit}
\section{Virtualisierung: Die Universale Turingmaschine}
