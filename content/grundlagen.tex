\section{Logisch-Mathematische Grundlagen}
Dieses Kapitel soll in konziser Weise die Ergebnisse der folgenden Fächer zusammenfassen,
die Voraussetzung für einen sicheren Umgang mit dem Methodenkoffer
der theoretischen Informatik sind:
\begin{itemize}
    \item Formale Logik (\autoref{al} und \autoref{pl})
    \item Mengenlehre (\autoref{mengenlehre} und \autoref{relationenUndFunktionen})
\end{itemize}
Jeder dieser Abschnitte ist knapp gehalten und hat einen Anreißer,
der es den Lesenden erlaubt schnell zu entscheiden,
ob eine Auffrischung vonnöten ist.
Die Abschnitte sind keine vollwertige Einführung in das jeweilige Gebiet,
hierfür sei v.a. \cite{link} empfohlen.

\subsection{Aussagenlogik}\label{al}
In diesem Abschnit werden die grundlegenden aussagenlogischen Junktoren
(Negation, Konjunktion, Disjunktion, Konditional, Bikonditional) eingeführt.\\


Die grundlegende Idee der Aussagenlogik ist,
die logische Verknüpfung zwischen Aussagen zu analysieren.
Entscheidend dabei ist die Art und Weise wie Aussagen kombiniert werden
und ob sie wahr oder falsch sind.
Aussagen kommen in der Aussagenlogik meist als großgeschriebene Variablen (A,B,C) vor.
Die folgenden Wahrheitsoperatoren (auch Junktoren genannt)
können genutzt werden um eine oder mehrere Aussagen zu kombinieren:

\begin{itemize}
    \item Die \emph{Negation} ist eine Umkehr des Wahrheitswertes einer Aussage,
wir schreiben formal: $\neg A$. Wenn $A$ wahr ist, ist $\neg A$ falsch und umgekehrt.
        \item Die \emph{Konjunktion} zweier Aussagen A und B (formal: $A \wedge B$)
            ist nur dann wahr, wenn A \emph{und} B wahr sind.
            Die Konjunktion ist daher auch als ``Boolesches Und'' bekannt.
        \item Die \emph{Disjunktion} zweier Aussagen A und B (formal: $A \vee B$)
            ist nur dann wahr, wenn A \emph{oder} B wahr sind.
            Die Disjunktion ist daher auch als ``Boolesches Oder'' bekannt.
        \item Das \emph{Konditional} zweier Aussagen A und B (formal: $A \rightarrow B$)
            ist nur dann \emph{falsch}, wenn A wahr ist und B falsch.
            Das Konditional wird auch (materiale) Implikation genannt.
        \item Das \emph{Bikonditional} zweier Aussagen A und B (formal: $A \leftrightarrow B$)
            ist nur dann wahr, wenn A und B den selben Wahrheitswert haben.
            Das Bikonditional wird auch Äquivalenz genannt.
\end{itemize}
Zusammenfassend kann man die Bedeutung der Junktoren \autoref{tab:al} entnehmen.

Das Konditional sorgt beim Einstieg in die Aussagenlogik erfahrungsgemäß für etwas Kopfzerbrechen.
Dies liegt vor allem daran,
dass das natursprachliche Äquivalent des Konditionals, die Wenn-dann-Sätze,
wesentlich mehr Bedeutung haben als die Aussagenlogik ausdrücken kann.\footnote{
    In \cite{link} 75-103 werden nicht weniger als sechs Aspekte diskutiert.
}
Für die Zwecke dieses Skriptes ist es am Besten,
das Konditional als Wahrheitfsfunktion zu verstehen,
die zwei Aussagen nach festen Regeln auf einen Wahrheitswert abbildet,
ohne dabei die Ausdruckskraft der natürlichen Sprachen zu erhalten.

Die Aussagenlogik ist
einerseits ein grundlegendes Werkzeug für alle formal arbeitenden Wissenschaften,
darunter auch die theoretische Informatik.
Andererseits sind einige Fragen der Aussagenlogik selbst Probleme,
die algorithmisch untersucht werden.
Zum Beispiel kann man sich fragen,
ob es für eine komplexe Aussagen-logische Formel mit vielen Junktoren und vielen Aussagenvariablen
eine ``Belegung''\footnote{
    Eine Belegung ordnet einer Aussagenvariable einen Wahrheitswert zu.
} dieser Variablen gibt,
welche diese Formel wahr (oder falsch) macht.
Gerade dieses Problem, in der Literatur SAT genannt,
ist zentral für die Komplexitätstheorie.

Die Inhalte dieses Abschnittes können in \cite{link} vertieft werden (Kapitel 2, 3, 7 und 11).

\begin{table}[ht]
    \caption{Wahrheitstabelle}
    \centering
    \begin{tabular}{|c|c|c|c|c|c|c|}
    \hline
          A
        & B
        & $\neg A$
        & $A \wedge B$
        & $A \vee B$
        & $A \rightarrow B$
        & $A \leftrightarrow B$
        \\
        \hline
          1
        & 1
        & 0
        & 1
        & 1
        & 1
        & 1
        \\
        \hline
          1
        & 0
        & 0
        & 0
        & 1
        & 0
        & 0
        \\
        \hline
          0
        & 1
        & 1
        & 0
        & 1
        & 1
        & 0
        \\
        \hline
          0
        & 0
        & 1
        & 0
        & 0
        & 1
        & 1
        \\

    \hline
    \end{tabular}
    \label{tab:al}
\end{table}


\subsection{Prädikatenlogik}\label{pl}

In diesem Abschnitt werden zentrale Konzepte der Prädikatenlogik
(ein- und mehrstellige Prädikation, Quantoren, Kombination von Quantoren)
vorgestellt.

Die Prädikatenlogik erweitert die Aussagenlogik,
indem sie Aussagen nicht mehr als gegebene Satzvariablen auffasst,
sondern das Werkzeug bereitstellt,
die interne Struktur von Aussagen zu analysieren.

Die grundlegende Struktur von Aussagen ist dabei die \emph{Prädikation},
eben die Aussage, dass 1-n Objekte in einem Verhältnis stehen.
Zum Beispiel würde die Aussage
``a ist ein Vokal'' formal mit $V(a)$ wiedergeben,
wobei $V(x)$ das Prädikat ``x ist ein Vokal'' darstellt
und ``a'' der Name eines Objekts ist.
$V(x)$ ist \emph{einstellig},
Prädikate können aber auch \emph{mehrstellig} sein:
$ABC(x,y)$ könnte zum Beispiel bedeutet: nach x kommt y im Alphabet.

Die wesentliche Analysetiefe bekommt die Prädikatenlogik aber dadurch,
dass sie es erlaubt, mit \emph{Quantoren} Existenz- und Allaussagen zu formalisieren.
Die Aussage ``Es gibt einen Vokal'' formalisieren wir mit $\exists x(V(x))$
und die Aussage ``Alle Vokale sind entweder a, e, i, o oder u.``:
\[\forall x (V(x) \rightarrow x = a \vee x = e \vee x = i \vee x = o \vee x = u)\]
Die Prädikatenlogik erlaubt es zudem Quantoren \emph{beliebig komplex zu kombinieren}:
\[\forall x \exists y(V(x) \rightarrow ABC(x,y))\]
Dies bedeutet, dass jeder Vokal einen Nachfolger im Alphabet hat,
was für Konsonanten eine falsche Aussage wäre (z ist der letzte Buchstabe).

Diese ``Schachtelung'' der Quantoren ist nicht nur eine technische Spielerei,
sie hat es ermöglicht, zentrale Ergebnisse der Mathematik präzise zu formulieren.\footnote{
    Ein Beispiel ist der Konvergenzbegriff, siehe \cite{link}, 10
}
Dies trifft auch auf wichtige Ergebnisse der theoretischen Informatik zu.
Wir werden in \autoref{pumping} das Pumping-Lemma als ein Beispiel hierfür kennenlernen.

Die Inhalte dieses Abschnittes können in \cite{link} vertieft werden
(Einleitung, 9-13 und Kapitel 4, 6, 8 und 12).
%Sowohl die Aussagenlogik als auch die Prädikatenlogik können in einem axiomatischen
%System formalisiert werden, dass es uns erlaubt aus grundg
%Axiomatisierung von AL und PL (\cite{link} 13)

\subsection{Mengenlehre}\label{mengenlehre}
In diesem Abschnitt werden die folgenden Inhalte erläutert:
\begin{itemize}
    \item Die Bedeutung grundlegender mengentheoretischer Begriffe\\
        (Menge, Element, leere Menge, Mächtigkeit, Extension, Universum)
    \item Wie man Mengen extensional und intensional spezifiziert
    \item Operationen auf Mengen (Vereinigung, Schnitt, Komplement)
    \item Der Begriff der Potenzmenge
\end{itemize}
\subsubsection{Grundlegende Begriffe}\label{grundmenge}
Unter einer \emph{Menge} verstehen wir eine Zusammenfassung von wohlunterschiedenen Objekten
zu einem Ganzen.\footnote{
    Dies ist eine Paraphrase der kanonischen Definition von Georg Cantor, siehe \cite{cantor}
}
Die ``wohlunterschiedenen Objekte'' nennt man auch die \emph{Elemente} einer Menge.
Der Begriff \emph{Extension} einer Menge bezeichnet alle Elemente einer Menge.
Mengen kann man \emph{extensional} spezifizieren,
indem man alle Elemente in geschweiften Klammern angibt,
zum Beispiel: $A = \{ 1,2,3 \}$.
Manche Mengen sind wohlbekannt,
zum Beispiel ist $\mathbb{N}$ die Menge der natürlichen Zahlen.
Da es unendlich viele natürlichen Zahlen gibt,
können wir die Extension nicht angeben
(alles Papier der Welt wäre nicht genug).
1,2,3 sind Elemente in beiden Mengen Menge.
Wir schreiben zum Beispiel $1 \in A$ und $1 \in \mathbb{N}$
um die Elementschaftsbeziehung zwischen Objekt/Element und Menge formal auszudrücken.
Die leere Menge ist eine spezielle Menge, die keine Elemente enthält,
wir bezeichnen die leere Menge mit $\emptyset$.
%Variablen für Mengen werde wir mit Großbuchstaben bezeichnen,
%Variablen für Elemente mit kleinen Buchstaben.

Die \emph{Mächtigkeit} einer endlichen Menge ist die Anzahl ihrer Elemente,
zum Beispiel ist die Mächtigkeit von $\emptyset$ null und von $A$ drei.\footnote{
    Die Mächtigkeit von unendlichen Mengen ist ebenfalls ein wohldefiniertes Konzept,
    das allerdings für die Zwecke dieses Skriptes nicht eingeführt werden muss.
    Siehe: \cite{link}, 15
}
Wir schreiben formal $|A| = 3$.
Mit dem Begriff \emph{Universum}\footnote{
    Im Englischen spricht man auch vom \emph{universe of discourse}
} bezeichnen wir die Gesamtheit der Elemente,
über die wir Aussagen treffen.
Zum Beispiel wären die natürlichen Zahlen ein Universum.
Man kann man Mengen auch \emph{intensional} angeben,
d.h.\ wir geben eine Bedingung an,
die alle Elemente eines Universums erfüllen müssen.
Für alle geraden natürlichen Zahlen wäre das z.B.:
\[\{n| n = 2m \wedge n,m \in \mathbb{N} \}\]
Links in der geschweiften Klammer wird eine Variable für die Elemente angegeben
(manchmal ergänzt durch die Spezifikation aus welchem Universum die Elemente ausgewählt werden).
Rechts wird die Bedingung angegeben, anhand derer Elemente ausgewählt werden.\footnote{
    Die ``naive'' Mengenlehre nimmt an,
    dass jegliche Bedingung auf der rechten Seite gewählt werden kann.
    Dies ist nicht der Fall, wie wir in \autoref{sec:paradoxien} genauer feststellen werden.
}
Auf der linken Seite kann eine komplexe Variable für Elemente auch detaillierter angegeben werden,
um auf der rechten Seite einfacher zu referenzieren:
$\{x + iy| x,y \in \mathbb{R}\}$.

Für Mengen gilt allgemein:
\[ \{a,b\} = \{b,a\} = \{a,b,a\} \]
Das heißt:
\begin{itemize}
    \item Die Elemente einer Menge haben keine Reihenfolge.
    \item Duplikate eines Elements in einer Extension werden ignoriert.
\end{itemize}



\subsubsection{Relationen zwischen und Operationen auf Mengen}

Mengen können in der \emph{Teilmengengenbeziehung} stehen,
zum Beispiel sind alle Elemente in A auch Elemente in $\mathbb{N}$.
Wir schreiben dann $A \subset \mathbb{N}$.
Wir schreiben $A \subseteq B$, wenn $A$ Teilmenge von $B$ ist oder $A = B$ gilt,
daher gilt auch $A \subseteq \mathbb{N}$.

Man kann Mengen unter Anderem mit folgenden Operationen kombinieren:
\begin{itemize}
    \item Vereinigung zweier Mengen A und B:\\
        $A \cup B := \{x|x \in A \vee x \in B\}$
    \item Schnitt zweier Mengen A und B:\\
        $A \cap B := \{x|x \in A \wedge x \in B\}$
    \item Komplement zweier Mengen A und B:\\
        $A \backslash B:= \{x|x \in A \wedge x \notin B\}$
%    \item Große Operationen (nötig)
%    \item Teilmengen
    \item Potenzmenge einer Menge A: $\mathcal{P}(A) := \{x|x \subseteq A\}$
\end{itemize}

Die Mächtigkeit der Potenzmenge über einer beliebigen Menge M ist stets:
$|\mathcal{P}(M)| = 2^{|M|}$.


Die Mengenlehre ist nicht nur die \emph{lingua franca}\footnote{Vergleiche \cite{link}, 14}
aller formal arbeitenden Wissenschaften,
sie ermöglicht es auch die grundlegenden Konzepte der theoretischen Informatik
konzise zu spezfizieren (siehe \autoref{sec:formalisierung}.
Der Inhalt dieses Unterkapitels kann in \cite{link}
in den Kapiteln 1.1, 9 und 10 vertieft werden.

\subsection{Relationen und Funktionen}\label{relationenUndFunktionen}
In diesem Abschnitt werden wir die folgenden Sachverhalte erläutern :
\begin{itemize}
    \item Tupel sind Mengen mit Reihenfolge.
    \item Relationen sind Mengen von Tupeln mit identischer Länge.
    \item Funktionen sind rechtseindeutige Relationen.
\end{itemize}
Dies erlaubt uns Funktionen als Menge von Tupeln zu begreifen,
was uns für endliche Funktionen
(wie z.B. eine Übergangsfunktion eines Automaten)
erlaubt,
die Funktion extensiv anzugeben (siehe voriger Abschnitt).

\subsubsection{Tupel}

Ein Tupel ist wie eine Menge eine Zusammenfassung von Elementen, nur gilt hier:
\[[a,b] \neq [b,a] \text{ und } [a,b] \neq [a,b,a] \text{ und } [b,a] \neq [a,b,a]\]
D.h. Tupel werden anstatt mit geschweiften mit eckigen Klammern angegeben,
haben eine Reihenfolge und daher können Duplikate nicht einfach ignoriert werden.

\noindent
Das Folgende sollten bekannt sein:
\begin{itemize}
    \item Ein Tupel mit zwei Elementen heißt geordnetes Paar.
    \item Ein Tupel mit drei Elementen heißt Tripel.
    \item Ein Tupel mit n    Elementen heißt n-Tupel.
    \item Das kartesische Produkt zweier Mengen A und B wird mit $A \times B$ bezeichnet
          und ist definiert als: $\{[a,b]|a \in A \wedge b \in B\}$.
    \item Analog lässts sich das kartesische Produkt von mehr als zwei Mengen definieren.
\end{itemize}

\subsubsection{Relationen}

Eine Relation ist eine Menge von gleichlangen Tupeln.
Relationen (wie auch Funktionen) bringen Struktur (manchmal auch eine Ordnung)
in ein Universum.
Relationen kann man im Sinne von \autoref{pl} auch als mehrstelliges Prädikat auffassen.
Sei z.B.
\[
    \Sigma = \{a,b,c,...,z\}
\]
unser Universum, also das deutschsprachige Alphabet,
bestehend aus 26 kleinen Buchstaben.\footnote{
Ohne Beschränkung der Allgemeinheit ignorieren wir hier Umlaute und Großbuchstaben.}
Eine mögliche Relation ist die Folgende:
\[
    ABC = \{[a,b], [b,c], [c,d], \ldots, [x,y], [y,z]\}
\]
$ABC$ ist die Relation eines Buchstabens zu seinem Nachfolger
gemäß der Konvention in deutschsprachigen Ländern.
$ABC$ ist ein zweistelliges Prädikat, daher sind alle Elemente der Relation geordnete Paare.
Man kann das bestehen einer Relation zwischen zwei Elementen des Universums
auch in der Infix-Notation angeben, z.B.: aABCb (a steht in der ABC-Relation zu b).

Ein weiteres Beispiel ist:
\[
    \begin{array}{lllllll}
        <_{ABC} &
        = &
            \{[a,b], &
            [a,c], &
            \ldots, &
            [a,y], &
            [a,z],\\
        &
        &
        &
            [b,c], &
            \ldots, &
            [b,y], &
            [b,z], \\
        &
        &
        &
        &
            \ldots,\\
        &
        &
        &
        &
        &
        &
            [y,z] \}
\end{array}
\]
$<_{ABC}$ gibt die gesamte Ordnung der Buchstaben an (nicht nur den unmittelbaren Nachfolger).
Analog zu oben gibt es auch hier eine Infix-Notation,
die den mathematischen Sehgewohnheiten entspricht: $a <_{ABC} d$ oder $f <_{ABC} x$.

\noindent
Folgende Sachverhalte sollten bekannt sein:
\begin{itemize}
    \item Eine Relation $R$ ist symmetrisch, wenn gilt: $\forall x,y(xRy \leftrightarrow yRx)$.
    \item Eine Relation $R$ ist reflexiv, wenn gilt: $\forall x(xRx)$.
    \item Eine Relation $R$ ist transitiv,
        wenn gilt: $\forall x,y,z(xRy \wedge yRz \rightarrow xRz)$.
    \item Eine Relation, die symmetrisch, reflexiv und transitiv ist,
          heißt Äquivalenzrelation (z.B. ist $=$ eine Äquivalenzrelation).
\end{itemize}

\subsubsection{Funktionen}
Funktionen sind rechtseindeutige Relationen,
d.h. für den Fall einer zweistelligen Relation R:
\[
    \forall x,y (xRy \rightarrow \neg \exists z (z \neq y \wedge xRz))
\]
Intuitiv bedeutet das: Wenn x der Input der Funktion ist,
dann ist der Output y der Funktion eindeutig bestimmt
(es gibt keinen anderen Wert z der von y unterschieden ist,
der auch Output sein kann, wenn x gegeben ist).
Da der Output meistens rechts steht, sagt man auch die Relation ist \emph{rechts}eindeutig.
Die Relation $ABC$ aus dem vorangegangenen Abschnitt ist eine Funktion,
die Relation $<_{ABC}$ nicht.

Folgende Sachverhalte sollten bekannt sein:
\begin{itemize}
    \item Eine andere Bezeichnung für Funktion ist ``Abbildung''.
    \item ''Input'' einer Funktion wird auch Definitionsmenge genannt.
    \item ''Output'' einer Funktion wird auch Wertemenge genannt.
    \item Bei einstelligen Funktionen (also rechtseindeutigen, zweistelligen Relationen)
        ist die Definitionsmenge die Menge aller Elemente,
        die an der ersten Stelle des geordneten Paares stehen.
    \item Bei einstelligen Funktionen (also rechtseindeutigen, zweistelligen Relationen)
        ist die Wertemenge die Menge aller Elemente,
        die an der zweiten Stelle des geordneten Paares stehen.
    \item Bei mehrstelligen Funktionen muss eine Signatur angeben, was zur Definitions-
        bzw. Wertemenge gehört. Z.B. hätte die Funktion $between$,
        die sowohl Vorgänger als auch Nachfolger im Alphabet angibt die folgende Signatur:
        \[ between:
            \Sigma \backslash \{a,z\}
            \rightarrow
            \Sigma \backslash \{z\} \times \Sigma \backslash \{a\} \]
\end{itemize}

Ein grundlegendes formales Verständnis von Relationen und Funktionen ist besonders wichtig
bei der Frage, wie Maschinen feststellen können,
ob eine Relation zwischen zwei Objekten besteht
und noch vielmehr bei der Frage, wie Maschinen Funktionen berechnen können.
In \autoref{subsec:problemeSprachen} werden wir uns genau diesen Fragen widmen.

%\subsection{Beweisstrategien \& Beweisskizzen}\label{bew}
