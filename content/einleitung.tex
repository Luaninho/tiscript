\chapter{Einleitung}\label{einleitung}
%\section{Warum theoretische Informatik?}
Warum lohnt sich die Beschäftigung mit theoretischer Informatik?
Die kurze Antwort: Sie liefert einen wesentlichen Teil des Handwerkszeugs,
um die Potentiale und Grenzen der technischen Informationsverarbeitung zu verstehen.
Die lange Antwort soll dieses Skript geben.

Die Nützlichkeit eines Werkzeugs leuchtet dann ein,
wenn es hilft Probleme zu lösen.
Daher ist der Problembegriff im Mittelpunkt dieses Skripts.
Was also ist ein Problem im Sinne der (theoretischen) Informatik?
Bevor wir diese Frage grundlegend beantworten können,
ist es hilfreich ein paar Probleme kennen zu lernen,
wie sie als Aufgaben im IT-Alltag auftreten können.

\section{Probleme}
Aufgaben, die viele IT-Fachkräfte Tag für Tag bearbeiten,
werden oft als Userstories dokumentiert.
Eine Userstory wird in der agilen Softwareentwicklung genutzt um Anforderungen zu spezfizieren.
Eine Userstory ist typischerweise in diesem Format angegeben:
''Als \textless Rolle\textgreater\ möchte ich \textless Feature\textgreater,
um \textless Ziel\textgreater\ zu erreichen.''
Diese Beispiele sollen uns im Skript begleiten:

\begin{itemize}
    \item \textbf{EVEN}: Als Kommisionier:in
         möchte ich die Anzahl der geplanten Paletten pro LKW kennen,
         um zu sehen, ob Platz für den Hubwagen bleibt. 
% TODO: Einkommentieren und Bearbeiten, wenn das jeweilige Kapitel geschrieben ist.
%        kennen, um meine Lieferungen zu optimieren.
%    \item \textbf{MATCH}: Als Einkaufsmanager:in möchte ich,
%        dass nur valide Lieferanten-IDs gespeichert werden
%        (beginnend mit einem ``L'', gefolgt von einer Zahl),
%        um Fehler in der Datenbank zu vermeiden.
%    \item \textbf{HALT}: Als Product Owner des Serverless-Computings möchte ich,
%        dass nur terminierende Code-Snippets deployt werden,
%        um die verfügbaren Ressourcen effizient zu nutzen.
%    \item \textbf{ROUTE}: Als Kommissionierer:in möchte ich,
%        dass morgens die optimalen Routen für die LKWs in die 24 Filialen berechnet werden,
%        um die Lieferzeit zu minimieren.
%    \item \textbf{QUANTUMSEC}: Als Chief Security Officer möchte ich,
%        dass die firmen-interne Kommunikation quantensicher verschlüsselt wird,
%        um die Vertraulichkeit der Absprachen zu garantieren.
\end{itemize}

Mit den Methoden der theoretischen Informatik ausgerüstet,
kann man einiges über diese Alltags-IT-Probleme sagen:
\begin{itemize}
    \item Mindestens ein Problem ist nach aktuellem Stand sehr schwer zu lösen.
    \item Mindestens ein Problem ist sehr einfach zu lösen.
    \item Mindestens ein Problem ist nicht lösbar.
    \item Bei mindestens einem Problem ist unklar, ob und wie es sich lösen lässt.
\end{itemize}

Die Methoden und Ergebnisse der theoretischen Informatik erlauben es,
diese und ähnliche Probleme auf einen Blick einzuteilen:
leicht lösbar, schwer lösbar oder unlösbar.

Eine Userstory ist jedoch keine geeignete Form,
um besagte Methoden anzuwenden.
Stark vereinfacht sind die Grundwerkzeuge der theoretischen Informatik
\emph{Abstraktion} und \emph{Formalisierung}.
Wie beide auf unsere Probleme anzuwenden sind,
soll in den folgenden Abschnitten exemplarisch dargelegt werden.

\section{Abstraktion}

Ein grundlegendes Instrument der Informatik (nicht nur der Theoretischen) 
ist die \emph{Abstraktion},
also die ausschließliche Betrachtung der Aspekte eines Sachverhalts,
die für einen gegebenen Zweck wesentlich sind.
Die Abstraktion ist der Weg von einer sehr speziellen Aufgabenbeschreibung,
wie einer Userstory, zu einem abstrakten Problem,
dessen Lösung in vielen spezifischen Kontexten anwendbar wird.
Mit etwas Glück ist diese Lösung bereits frei verfügbar implementiert
und daher leicht nachnutzbar.

Es gibt leider keinen einfachen, eindeutigen oder automatisierbaren Weg
duch Abstraktion von einer Userstory zu einem abstrakten Problem zu kommen.
Dafür sind intellektuelle Kreativität
und ein gutes Verständnis des vorliegenden Zwecks notwendig.
Ein möglicher Fehler ist zum Beispiel,
zu wenige Aspekte auszulassen,
so dass das abstrakte Problem nicht so einfach wie möglich ist. 
Ebenso ist es ein Fehler, zu viele Aspekte auszulassen,
so dass das abstrakte Problem nicht so detailreich wie nötig ist,
um dem vorliegenden Zweck gerecht zu werden.
Unser vorliegender Zweck ist die Beantwortung der Frage,
was ein Problem im Sinne der theoretischen Informatik ist.\footnote{
    Diesen muss man von den Zielen unterscheiden,
    derentwegen die Userstories formuliert wurden.
    Diese beiden Zwecke können sich überschneiden,
    sind aber voneinander zu unterscheiden.
}
So könnte eine Abstraktion von \textbf{EVEN} für diesen Zweck aussehen:
\begin{enumerate}
    \item \emph{Als Kommisionier:in
         möchte ich die Anzahl der geplanten Paletten pro LKW kennen,
         um zu sehen, ob Platz für den Hubwagen bleibt.
        } (ursprüngliche Userstory).
    \item \emph{Identifikation der Anzahl an Paletten, um verfügbaren Platz zu quantifzieren.}
        (Wer die Anzahl kennen will und wofür der Platz gedacht ist,
        ist für unseren Zweck gleichgültig.)
    \item \emph{Ist die Anzahl an Paletten (un)gerade?}
        (Es ist z.B. egal,
        ob zwei, drei oder mehr Paletten pro LKW nebeneinander passen,
        Hauptsache es bleibt Platz.)
    \item \emph{Ist eine Zahl gerade?}
        (Es ist egal, ob wir dies für Paletten oder andere Objekte entscheiden.)
\end{enumerate}

Wie wir im Laufe des Skriptes sehen werden, ist die letzte Formulierung des Problems die 
wesentliche Vorstufe, um die Methoden der theoretischen Informatik anzuwenden.\footnote{
Es ist natürlich möglich,
dass dieses abstrakte Problem nicht das richtige Resultat für einen anderen Zweck ist.
Für eine Kommissionierer:in mag die Anzahl noch aus anderen Gründen interessant sein,
z.B. um die Auslastung der LKWs zu bewerten und zu optimieren.
Wie man Abstraktionen im Sinne der Software Entwicklung einsetzen kann,
lehrt die Disziplin des Software Engineerings.}
Wir kommen so auf den Kern des \textbf{EVEN}-Problems:
\begin{itemize}
    \item \textbf{Gegeben:} Eine Zahl
    \item \textbf{Gesucht:} Ein Wahrheitswert, der anzeigt, ob die Zahl gerade ist. 
\end{itemize}
Diese Form der Problemangabe (Gegeben/Gesucht) wollen wir \emph{informell} nennen.
Dabei ist für die informelle Angabe wichtig,
sowohl den Typ des (potentiell komplexen) Eingabewerts (hier: eine Zahl),
als auch des (potentiell komplexen) Ausgabewerts (hier: ein Wahrheitswert) zu spezifizieren.
Mit der informellen Problemangabe lassen sich bereits drei sehr grundlegende Fragen stellen:
\begin{enumerate}
    \item Wie muss \textbf{EVEN} dargestellt sein, damit eine Maschine es lösen kann?\\ 
        In anderen Worten: Wie kann \textbf{EVEN} \emph{formalisiert} werden? 
    \item Wie effizient kann eine Maschine \textbf{EVEN} lösen?\\
        In anderen Worten: Wie \emph{komplex} ist \textbf{EVEN}?
    \item Kann eine Maschine \textbf{EVEN} überhaupt lösen?\\
        In anderen Worten: Ist \textbf{EVEN} \emph{berechenbar}?  
\end{enumerate}
Diese Fragen können wir natürlich nicht nur für \textbf{EVEN} stellen.
Auch die anderen Userstories von oben lassen sich in eine informelle Form bringen,
die diese Art der Fragestellung erlaubt.
Drei grundlegende Teildisziplinen der theoretischen Informatik
geben Antwort auf diese Fragen:
\begin{enumerate}
    \item Die \textbf{Theorie der formalen Sprachen und Automaten}
    \item Die \textbf{Komplexitätstheorie}
    \item Die \textbf{Berechenbarkeitstheorie}
\end{enumerate}

Der Rest des Skriptes wird
in einige der grundlegenden Ergebnisse dieser drei Teildisziplinen einführen,
ihre Wechselwirkung beschreiben
und die Anwendung ihrer Methoden auf die eingangs erwähnten Userstories erläutern.
Anhand der oben aufgezählten Fragen wird klar,
dass hier großes Potential für die IT-Fachkräfte besteht,
die unnötige Arbeit vermeiden
und die besten der bekannten Lösungen für ihre Probleme implementieren wollen. 

Was sind aber nun Probleme generell im Sinne der theoretischen Informatik?
Bisher haben wir nur einzelne Userstories aufgezählt
und eine davon als informelles Problem angegeben.
Diese Angabe legt nahe,
dass wir für die vielen Anforderungen, die wir mit Userstories dokumentieren,
eigentlich stets eine \emph{Funktion} suchen,
also eine Abbildung aus einer Definitionsmenge - oder dem Input
in eine Wertemenge - dem Output.
Da wir uns noch nicht im Bereich des formalen Arbeitens bewegen
(das passiert im nächsten Abschnitt),
können wir die Zuordnung Problem $\leftrightarrow$ Funktion nur postulieren
und hoffen, dass sie einleuchtet.
Bis jemand eine alternative und bessere Erläuterung des Problembegriffs liefern kann,
wollen wir uns damit zufrieden geben.

\section{Formalisierung}\label{sec:formalisierung}

Warum sind die philosophischen Erwägungen des vorigen Abschnittes für die Informatik relevant?
Zum einen wollen wir Aussagen darüber treffen,
wie und ob eine beliebige Maschine ein gegebenes Problem lösen kann. 
Zum anderen betrachten wir in der theoretischen Informatik oftmals nicht unbedingt nur einzelne Probleme,
sondern Gruppen von Problemen und deren Beziehungen zueinander.
Beide Gründe resultieren in Folgefragen:
\begin{itemize}
\item Wie muss ein Problem formalisiert sein,
      dass wir Aussagen über das Problem unabhängig von einer konkreten Maschine treffen können?
\item Wie kann man den ``Raum'' der Probleme beschreiben?
\end{itemize}

Zur ersten Frage stellen wir fest,
dass wir uns bei den gesuchten Aussagen nicht auf eine bestimmte Computer-Architektur
oder eine bestimmte Programmiersprache einlassen sollten:
Die Aussage, dass ein Problem \emph{un}berechenbar ist,
also, dass es keine Maschine gibt, die das Problem lösen kann,
wäre nicht abschließend lösbar:
es müsste für die unendliche Menge an Kombinationen von Hardware-Setups und Programmiersprachen
gezeigt werden.
Auch die Frage nach der Komplexität eines Problems wäre schwer zu erörtern:
Kann es nicht sein, dass das Problem auf dieser Hardware
und mit jener Programmiersprache nicht doch wesentlich effizienter zu lösen ist?
Wir wollen die Probleme also so abstrakt formalisieren,
dass unsere Aussagen über sie prinzipiell auf \emph{allen} Computerarchitekturen
und für \emph{alle} Programmiersprachen gültig sind.
Wir suchen also eine Formalisierung,
die von allen unwesentlichen Aspekten von Hardware und Programmierung abstrahiert.

Die zweiten Frage - wie man den ``Raum'' der Probleme beschreiben kann -
bedingt ebenfalls eine abstrakte Repräsentation des Problems.
Nur so können wir das formale Handwerkszeug der Mengenlehre anwenden,
das uns eine konzise und formal einwandfreie Beschreibung dieses ``Raumes'' erlaubt.\footnote{
dann könnten wir auch die Anführungszeichen weglassen, weil wir uns dann einer formalen Definition des Raumes
bedienen könnten.}
Die erste Möglichkeit bestünde darin,
Probleme einfach als ``Atome'' aufzufassen,
also als Einheiten ohne innere Struktur,
auf denen wir dann Mengen bilden können.
Es ist allerdings gerade die innere Struktur von Problemen,
die zum Beispiel Abhängigkeiten zwischen Problemen klar werden lassen:
Zum Beispiel kann man das Problem der Multiplikation nur
als wiederholte Anwendung der Addition verstehen,
wenn wir etwas über die interne Struktur der Multiplikation aussagen.

Gesucht ist also eine grundlegende Art, Probleme zu formalisieren, mit den folgenden Bedingungen:
\begin{enumerate}
 \item Die Form hängt nicht von einer bestimmten Maschine oder Programmiersprache ab.
 \item Die Form erlaubt es, Probleme mengentheoretisch zu behandeln.
 \item Die Form erlaubt es, dass Aussagen über die innere Struktur der Probleme getroffen werden können.
\end{enumerate}

Wir werden sehen, dass eine Repräsentation von Problemen als formale Sprachen genau diese Bedingungen erfüllen.
Um dies besser fassen zu können, wollen wir die Formalisierung des \textbf{EVEN}-Problems beispielhaft durchgehen.

Wie bereits erwähnt, legt die informelle Angabe des \textbf{EVEN}-Problems nahe,
dass wir nach einer Möglichkeit suchen eine Funktion zu berechnen:
\begin{itemize}
    \item Der Definitionsbereich der Funktion ist die Menge aller natürlichen Zahlen,
        die wir wie gewohnt mit $\mathbb{N}$ bezeichnen wollen.
    \item Der Wertebereich der Funktion ist die Menge der Wahrheitswerte 
    in ihrer kanonischen Repräsentation als 0 und 1.\footnote{
    Hier haben wir natürlich die implizite Annahme getroffen,
    dass es zwei Wahrheitswerte ``gibt'' und nicht mehr (\emph{tertium non datur}).
    Das hat sowohl philosophische (TODO: Link), als auch formale Implikationen (TODO: Link),
    die wir geneigten Leser:innen zur Kontemplation anempfehlen.
    Für unsere Einführung sind diese Erwägungen aber nebenrangig.
}
\end{itemize}
Gesucht ist also $f_{EVEN}: \mathbb{N} \rightarrow \{0,1\}$.
Es gilt zum Beispiel:
\begin{itemize}
    \item Sei $n = 5$ gegeben, dann ist $[5,0] \in f_{EVEN}$.
    \item Sei $n = 4$ gegeben, dann ist $[4,1] \in f_{EVEN}$.\footnote{
            Wie in \ref{relationenUndFunktionen} angerissen,
            fassen wir Funktionen als Menge von rechtseindeutigen Tupeln auf.}
\end{itemize}

\noindent
Wie finden wir einen Mechanismus oder eine Maschine, die $f_{MAX}$ berechnet?

Die einfachste Lösung wäre eine große Tabelle, die uns für jedes $\nu$
das jeweilige Maximum $n$ ablesen lässt.
Das wäre zwar nicht sonderlich Zeit-effizient,
da wir viele Vergleiche anstellen müssten,
ob unser gegebenes $\nu$ mit einem Tabelleneintrag übereinstimmt,
aber immerhin wäre die Methode Zeit-effektiv, 
d.h. nach endlich vielen Schritten hätten wir ein Ergebnis.
Allerdings ist $\cal{N}$ so groß,
dass die angenommene Anzahl der Atome in unserem Universum nicht ausreicht,
um alle Einträge der Tabelle zu verwirklichen.
Die Methode ist also auch nicht Platz-effizient und nur dann Platz-effektiv,
wenn wir unendlich viel Platz zur Verfügung haben.

Ein weiterer Ansatz $f_{MAX}$ zu berechnen,
wäre eine Maschine zu erfinden, der wir $\nu$ übergeben,
und die uns $n$ zurückgibt.
Es wäre demnach zu zeigen, ob die Maschine die Aufgabe in endlicher Zeit
und mit endlichen Platz-Ressourcen bewältigen kann.
%Wir könnten die Maschine von der Bedienung her noch ein Stück einfacher gestalten
%und ihr direkt $\nu$ und $n$ übergeben und eine Lampe oder ein Schalter 
%(oder eine andere binäre Ausgabe) zeigt an, ob $[\nu, n] \in f_{MAX}$.
Nehmen wir in einem Gedankenexperiment an, wir hätten eine solche Maschine erfunden.
Wie müssten wir den Input und Output der Maschine gestalten,
d.h. in welche Form müssten wir das \textbf{MAX}-Problem bringen,
dass die Maschine es lesen kann und wir den Output wieder verstehen können? 

Ein zugängliches Modell für den Input ist die Tastatur und der Bildschirm:
Eine endliche Menge von Zeichen (angezeigt durch Drücken der Knöpfe einer Tastatur)
werden von der Maschine übersetzt in eine Reaktion der Maschine 
(angezeigt durch die Pixel einer digitalen Ausgabe).
Tatsächlich werden die meisten Informationen,
die eine Maschine oder ein Computer verarbeiten,
intern in eine sehr kleine Menge an Zeichen übersetzt:
in Zeichenfolgen bestehend aus Nullen und Einsen.
Tastaturen und Bildschirme machen uns die Interaktion mit der Maschine einfacher,
allerdings sind sie für die Maschine selbst nicht nötig
(unsere Tastatur könnte also nur drei Knöpfe beinhalten: 0, 1 und die Enter-Taste,
die Ausgabe könnte ein Drucker sein, der 0 und 1 auf ein Papierband schreibt).

%IT-Systeme kodieren Zahlen, Videos, Bilder, Musik, Texte, im Allgemeinen alle Datenstrukturen
%und Ausgabewerte als Folgen von Nullen und Einsen,
%also als binäre Zeichenfolgen.
%Ebenso werden Anweisungen, was mit diesen binären Daten passieren soll,
%als binäre Zeichenfolgen kodiert.

Ganz allgemein könnte man also sagen, dass alle Probleme, die ein Computer lösen soll,
sich in diese Form bringen lassen:
\begin{itemize}
    \item \textbf{Gegeben:} Eine binäre Zeichenfolge.
    \item \textbf{Gesucht:} Eine binäre Zeichenfolge.
\end{itemize}
%Oder eben gegeben als Funktion $f: \mathcal{P}(\{0,1\}) \rightarrow \{0,1\}$
Übertragen auf \textbf{MAX} müssen wir eine binäre Repräsentation von $\nu$ und $n$ finden.
Damit wir dies auf eine methodische Art und Weise erreichen
und die gefundenen Methoden später von \textbf{MAX} auf andere Probleme übertragen können,
müssen wir zunächst noch einige Begriffe definieren und Konventionen einführen.

\subsection{Zeichen}

Eine Menge von Zeichen bezeichnen wir als \emph{Alphabet}, kurz: $\Sigma$.
Zum Beispiel wäre das Alphabet für die meisten IT-Systeme: $\Sigma = \{0,1\}$.\\

\noindent
Wir verwenden diese Variablen für Zeichen: $a,b,c$;
wenn wir diese Buchstaben verwenden und nichts anderes angeben,
gilt implizit: $a, b, c \in \Sigma$ für ein gegebenes $\Sigma$.\footnote{
Wenn $\Sigma$ nicht explizit angegeben oder aus dem Kontext herleitbar ist,
wird $\Sigma$ also auch als Variable für (irgend) ein Alphabet verwendet.}\\

\noindent
Eine Operation auf den Zeichen ist die \emph{Konkatenation},
also die Verbindung von zwei Zeichen, z.B. $01$.
Wie bei der Multiplikation mit Variablen in der Mathematik üblich,
wird die Konketanation nicht explizit mit einem eigenen Zeichen repräsentiert.

\subsection{Worte}
Das Ergebnis einer beliebig oft ausgeführten Konkatenation,
also eine (binäre) Zeichenfolge,
bezeichnen wir als \emph{Wort}, z.B.: $0101010001001001$\\

\noindent
Die Menge aller Wörter,
die sich aus einem Alphabet bilden lassen,
bezeichnen wir als \emph{Kleensche Hülle}, dargestellt als Stern: $\Sigma^*$.\\

\noindent
Wir verwenden diese Variablen für Worte:
$u, v, w, x, y, z$; wenn wir diese Buchstaben verwenden und nichts anderes angeben,
gilt implizit: $u, v, w, x, y, z \in \Sigma^*$.\\

\noindent
Die \emph{Länge} eines Wortes ist die Anzahl seiner Zeichen.
Formal benutzen wir die Betragsstriche um ein Wort, um seine
Länge zu bezeichnen. Wenn z.B. $w = 0101010001001001$, dann gilt $|w| = 16$.\\

\noindent
Das Wort mit der Länge 0, also das Wort, dass aus null Zeichen besteht,
nennen wir das \emph{leere Wort}, kurz: $\epsilon$.
Es gilt $|\epsilon| = 0$.\\

\noindent
Für die n-malige Wiederholung eines Zeichens oder Wortes führen wir den \emph{Wiederholungsoperator}
ein, eine hochgestellte Zahl $n \in \mathbb{N}$: $a^n$ oder $w^n$.
Z.B. $0^3 = 000$ und $(10)^2 = 1010$.
$a^0$ bzw. $w^0$ entspricht dabei $\epsilon$.\\

\noindent
Wollen wir ein angeben, dass ein Zeichen oder Wort beliebig oft (also auch 0-mal) wiederholt
wird, nutzen wir dafür einen hochgestellten Stern, den \emph{Kleene-Operator} $a^*$ oder $w^*$.
Mit dem Kleene-Operator bezeichnen wir immer eine unendliche Menge von Wörtern, z.B.:
$a^* = \{\epsilon, a, aa, aaa, aaaa, \dots\}$

\subsection{Formale Sprachen}

Nachdem wir von Zeichen und Worten gesprochen haben, liegt die Frage nahe,
ob man denn auch von Sprachen sprechen kann und tatsächlich ist die formale Sprache 
ein zentraler Begriff der theoretischen Informatik.
Bevor wir den Begriff der formalen Sprachen einführen, ist noch eine Warnung angebracht:
Während die Verwendung des Begriffs ''Zeichens''
noch recht intuitiv mit dem Begriff des Buchstaben zusammenfällt
(zumindest im deutschsprachigen Raum),
ist schon der formal eingeführte Wort-Begriff eher eine Zumutung für unsere Intuition:
Worte im Sinne des vorangegangenen Abschnitts sind einfach irgendwelche Zeichenfolgen,
egal ob sie aussprechbar oder im Kontext einer natürlichen Sprache
(wie deutsch oder englisch) einen Sinn ergeben.

Man müsste also eigentlich das natursprachliche $Wort_N$ von dem formalen $Wort_F$ trennen,
das wir oben eingeführt haben.
Eine Crux aller formal arbeitenden Wissenschaften ist,
dass die Menge an Zeichen und Begriffen begrenzt ist und daher Doppeldeutigkeiten auftreten.
Wir müssen also durch Konventionen klären, welche Bedeutung wir einem Begriff geben wollen:
Im Kontext dieses Skriptes bezeichnen wir mit dem Begriff Wort die Zeichenfolge, also $Wort_F$.
Sollte dies nicht der Fall sein und der Kontext nicht ausreichen,
um zwischen beiden Bedeutungen zu unterscheiden,
werden wir die subskribierten Varianten verwenden.

Die vorige Überlegung trifft nicht nur auf Worte, sondern auch auf \emph{Sprachen} zu.
Eine Sprache im Sinne der theoretischen Informatik ist einfach eine Menge von Worten.
Z.B. ist $L = \{0, 1, 00, 01, 10, 11 \}$ die Sprache der Worte,
die aus einem oder zwei Binärzeichen bestehen.
Wesentlich ist eine formale Sprache als \emph{Menge} im Sinne der Mengenlehre aufzufassen, 
nicht als Kommunikationsmedium mit einer eigenen Sprachgeschichte und Sprachgemeinschaft.\\

\noindent
Wir verwenden $L$ (ggf. mit unterschiedlichen Subskripten) als Variable für Sprachen;
wenn wir $L$ verwenden und nichts Anderslautendes angeben ist,
gilt implizit $L \subseteq \Sigma^*$ für ein gegebenes $\Sigma$.\\

\noindent
Da Sprachen Mengen sind,
sind alle mengentheoretischen Operationen\footnote{siehe \autoref{mengenlehre}}
auf Sprachen definiert,
z.B. bezeichnet $L_1 \cap L_2 = \{w | w \in L_1 \wedge w \in L_2\}$
die Schnittmenge zweier Sprachen, also alle Wörter,
die sowohl Element von $L_1$ als auch von $L_2$ sind.

\subsection{Probleme als Sprachen}\label{subsec:problemeSprachen}
Mit den oben eingeführten Begriffen können
wir nun ein Problem wie \textbf{MAX} im Sinne der theoretischen Informatik
\emph{formalisiert} spezifizieren.
Die grundlegende Idee dabei: Ein Problem lässt sich durch eine formale Sprache repräsentieren. 
Die formale Sprache besteht dabei aus der binären Repräsentation genau der Paare $\nu$, $n$,
für die gilt $[\nu, n] \in f_{MAX}$. 
Die Sprache ist also der großen Tabelle nicht unähnlich,
in der man die ``richtigen'' $n$-Werte ablesen kann.
Wir wissen, dass diese Sprache unendlich viele Wörter hat und dass es unwirtschaftlich wäre
die Sprache wirklich aufzuschreiben (wir würden nie fertig werden).
Perspektivisch müssen wir also nach endlichen Methoden suchen,
(potentiell) unendliche Sprachen zu verarbeiten.

Wie genau können wir aber ein Problem wie \textbf{MAX}
als Menge binärer Zeichenfolgen darstellen?
Bisher haben wir ja mit $\cal{N}$ und $\mathbb{N}$ mehr als zwei Zeichen genutzt:
also $\Sigma = \{0,1,2,3,4,5,6,7,8,9,[,],``,''\}$
Dafür müssen wir zunächst eine Binärkodierung für Zahlen finden:
$Bin_{LE}: \mathbb{N} \rightarrow \{0,1\}$
Die Little Endian Dual-Kodierung erfüllt diese Anforderung,
also die Repräsentation von natürlichen Zahlen bei der
das niedrigstwertige Bit am Anfang steht.
Zum Beispiel ist $Bin_{LE}(6) = 011$, mit 
$
0 \cdot 2^0
+ 1 \cdot 2^1
+ 1 \cdot 2^2
= 6
$.\footnote{TODO: Füge Ressource hinzu, die Dual-Kodierung und LE/BE erklärt;
entweder als Referenz oder in Kapitel 1}

Mit $Bin_{LE}$ können wir $\Sigma$ auf die folgende Menge reduzieren: $\{0, 1, [, ], ``,''\}$.
Damit wir nun nicht nur einfache Zahlen, sondern auch Zahlenfolgen kodieren können,
müssen wir noch eine weitere Kodierung einführen,
die es erlaubt an einer binärem Wort zu erkennen,
wo eine Zahl anfängt und die nächste beginnt.
Dies kann mit der folgenden Kodierung erreicht werden: 
$Bin_{TUPLE}:$ 

\begin{itemize}
    \item $Bin_{TUPLE}(0) = 00$
    \item $Bin_{TUPLE}(1) = 01$
    \item $Bin_{TUPLE}(,) = 11$
\end{itemize}
Z.B. ist $Bin_{TUPLE}(11,10,01,00) = 0101$ $11$ $0100$ $11$ $0001$ $11$ $0000$.
Damit haben wir $\Sigma$ auf vier Zeichen reduziert: $\{0,1,[,]\}$.
Wir könnten nun eine geeignete binäre Repräsentation von Tupeln finden,
um beliebige Tiefen in der Schachtelung zu realisieren.
Allerdings gibt es eine einfachere Möglichkeit: Konvention.
Wir legen fest, dass die letzte binär kodierte Zahl $n$ entspricht,
alle anderen binär kodierten Zahlen gehören zu $\nu$.

Für die Sprache $L_{MAX}$, die das Problem \textbf{MAX} kodiert oder auch repräsentiert,
können wir also folgende Aussagen treffen:
\begin{itemize}
    \item $0100$ $11$ $0101$ $11$ $0101$ $\in L_{MAX}$, da $f_{MAX}([[2,3]) = 3$
    \item $0100$ $11$ $0101$ $11$ $0100$ $\notin L_{MAX}$, da $f_{MAX}([[2,3]) \neq 2$
\end{itemize}

Sind nun Probleme identisch mit den formalen Sprachen, die sie repräsentieren? 
Die Antwort auf diese Frage hat sehr viele philosophische Implikationen,
die wir nicht im Rahmen dieses Skriptes behandeln können.\footnote{Sprachphilosophisch
könnten wir uns fragen, ob die Übersetzung eines Problems in eine formale Sprache
überhaupt eindeutig und verlustfrei möglich ist.
Oder ob es überhaupt so etwas wie Probleme gibt, die von ihrer repräsentierten Sprache 
getrennt existieren.}
Fakt ist zum einen, dass das \emph{Wortproblem}, also ob für ein Wort $w$ und eine Sprache $L$
gilt $w \in L$,
eng verknüpft ist mit einem abstrakten Problem, das von der formalen Sprache repräsentiert wird. 
Fakt ist ebenfalls, dass in vielen Textbüchern die formale Sprache und ihr Problem
im Prinzip miteinander austauschbare Entitäten sind. 
Wir wollen allerdings sorgsam sein und die beiden voneiander trennen.\\
%Wir wollen in unserem Skript allerdings davon sprechen, dass eine Sprache ein Problem
%repräsentiert, ohne eine Aussage zu treffen,
%ob die beiden identisch oder voneinander zu unterscheiden sind,
%da wir uns vorgenommen haben das Potential der theoretischen Informatik für das Lösen
%von IT-Problemen zu bewerten.
%Die spannenden philosophische Diskussionen lassen wir hier aus. 

\noindent
Um von unserer Userstory zu $L_{MAX}$ zu gelangen,
haben wir einen beträchtlichen formalen Aufwand investiert.
Daher stellt sich die Frage: Ist dieser Aufwand gerechtfertigt? 
Warum also geben wir Probleme als Wortprobleme von formalen Sprachen an?
Dies sind die wesentlichen Antworten auf beide Fragen,
nach Wichtigkeit aufsteigend sortiert:
\begin{itemize}
    \item Abstrakte Probleme mit Sprachen zu identifizieren gibt uns eine
        konzise, unmissverständliche Beschreibung von Problemen,
        die auch von Maschinen verarbeitet werden kann. 
    \item Die Abstraktion die in der Formalisierung wiedergespiegelt ist,
        ermöglicht sehr generelle Aussagekraft über das speziell betrachtete Problem hinweg.
    \item Aussagen über (effiziente) Berechenbarkeit werden mit den Methoden der Mathematik
        beweisbar, da wir u.a. die Ergebnisse der Mengenlehre, der Arithmetik
        und der Analysis anwenden können.
\end{itemize}

Zusammenfassend können wir feststellen:
\begin{itemize}
    \item Die theoretische Informatik arbeitet mit Abstraktion und Formalisierung.
    \item Probleme lassen sich als Sprachen repräsentieren.
        Ist es lösbar, ob ein Wort zu einer Sprache gehört,
        ist auch das repräsentierte Problem lösbar. 
    \item Die abstrakten Probleme der theoretischen Informatik lassen sich mit
        mathematischen Werkzeugen bearbeiten,
        wodurch ihre Ergebnisse theoretisch gefestigt sind 
        und die Aussagekraft sehr generell ist.
\end{itemize}
